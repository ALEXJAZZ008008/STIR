
\batchmode
\documentclass{article}
\usepackage{a4wide}
\usepackage{setspace}
\usepackage{longtable}
\usepackage{hyperref}
\def\R2Lurl#1#2{\mbox{\href{#1}{\tt #2}}}                                              
\newcommand{\tab}{\hspace{5mm}}
% attempt to define lower levels than subsubsections
% This does not work very well as you need to put them in extra {}
% to avoid the fonts being messed up etc.
% However, it does work for automatic numbering
\newtheorem{subsubsubsection}{{}}[subsubsection] 
\newtheorem{subsubsubsubsection}{{}}[subsubsubsection] 

% a simple definition for formatting command lines
\newcommand{\cmdline}[1]{\par \noindent $>$ \texttt{#1}\par}

\begin{document}


\begin{spacing}{2}
\begin{center}

\textbf{
{\Huge  STIR} 
\huge
\\[1cm]
\textit{ Software for  Tomographic \\ Image Reconstruction}
}
\\[3cm]

\textbf{{\huge User's Guide\\
 Version 2.1}}
\end{center}

\end{spacing}

\large 

\noindent 
K. Thielemans (Hammersmith Imanet Ltd)\\
D. Sauge, C. Labb\'e, C. Morel (Hopital Cantonal Gen\`eve)\\
M. Jacobson (Technion University)\\
A. Zverovich (Brunel University) \\
T. Beisel (Univ. of Paderborn) \\
Ch. Tsoumpas (Hammersmith Imanet and Imperial College London)



\newpage

\tableofcontents


\section{
Introduction}

The objective of this document is to give practical information 
about the use of the object-oriented library for 3D Reconstruction 
of PET data, called \textit{STIR}. The most recent version of this 
document (and the library) can be found on \\
\R2Lurl{http://stir.sourceforge.net}{http://stir.sourceforge.net}.


This library was originally developed by the (now finished) PARAPET 
project (funded by the European Union), extended by Hammersmith 
Imanet and made into an Open Source project. \\
The current library has different license restrictions than the 
original PARAPET distribution. Details on licensing are to be 
found during the registration process, and the file \textbf{STIR/LICENSE.txt} 
that comes with the distribution. 


Pending a publication on the current library, please put references 
to {[}Thi06{]} in your own publications (for \textit{STIR} 1.x, use [Lab99a], [Lab99b]).



This guide attempts to give end users the shortest pathway to 
perform 3D reconstructions using \textit{STIR}. It mainly provides 
procedures for downloading and installing the necessary gnu g++ 
compiler tools and the \textit{STIR} reconstruction building blocks. 
Then a brief explanation is given about how to run reconstruction 
algorithms as well as additional utilities. A description of 
the reconstruction building block library can be found on our 
web site (section \textit{documentation}). A short review of analytic 
and iterative reconstruction methods is available in [PAR1.3] 
(section \textit{Dissemination}).

\section{
A general note on documentation in STIR}

Although we attempt to keep all documentation up-to-date, we 
recommend to read documentation in the following order:\\
1.\tab 
general overview documents, such as this User's Guide\\
2.\tab 
online generated documentation (produced by doxygen). This is 
probably more up-to-date than the previous category, as it is 
produced from comments and (partly) code in the source files.\\
3.\tab 
check the source itself if in any doubt


Any comments on documentation, and especially contributions are 
always welcome. Please use the stir-users mailing list.

\section{
Installation}

\subsection{
Installing source files}

Download the source at the \textit{STIR} site: 
\begin{center}
\R2Lurl{http://stir.sourceforge.net }{http://stir.sourceforge.net} 
\end{center}
\noindent (Section \textbf{Registered Users}).\\
Download the source files in the zip format and use unzip from \\
\R2Lurl{http://www.info-zip.org/pub/infozip/}{http://www.info-zip.org/pub/infozip/}. 
Extract using
\cmdline{unzip -a file.zip}


The \texttt{-a} option makes sure that files are extracted using the 
end-of-line convention appropriate for your system. Note that 
other programs that can handle zip files (such as WinZip) should 
work as well, although you might have problems with the EOL convention.


Note that you can put the distribution in any location (it does 
not have to be your home directory).


The result is a \textbf{STIR} directory and subdirectories, described 
in the annex section.

\subsection{
Installing external software}
If you are running Ubuntu or Debian, you should be able to get do this using

\cmdline{apt-get install  gcc g++ make libncurses-dev libX11-dev libboost-dev tcsh}

(prefix with \texttt{sudo} for Ubuntu). For other systems, details are given below.

\subsubsection{Download/install boost}
\textit{STIR} uses the well-respected \textit{boost} library. This library
is currently installed in many systems (especially on Linux). If it
is not, you can download it from 
\R2Lurl{http://www.boost.org}{http://www.boost.org}. Currently \textit{STIR} only
uses the include files from \textit{boost} (so you do not need to build
the boost libraries). However, you will need to tell your compiler
where to find the \texttt{boost/} directory with all the include files.
See the Compilation section for some info (use \textit{make} variable 
\textit{EXTRA\_CFLAGS}).

\subsubsection{
C++ Compiler and make}

In order to compile and run the \textit{STIR} programs, we currently recommend 
using a recent version of the GNU gcc compiler, at least version 
3.0.1. The final release of \textit{STIR} 2.1 should work with gcc 4.2 up to 4.5.
Check the mailing lists for more info.
\footnote{{\small A few years ago, some work-arounds were put in place to 
keep \textit{STIR} compatible with gcc version 2.95.2, but this version is unlikely to
to be usable for recent versions of \textit{STIR}. It is known that with this
old version, some functionality is not supported. Specifically, 
some programs that take a parameter file as input also list the 
actual parameters used on stdout. This is especially useful if 
you used the interactive version and want to copy this output 
to a file. There is a bug in gcc 2.95.2 though such that this 
info cannot be listed correctly for the hierarchies based on 
registries (pretty much everything in section 
\ref{sec:user-selectablecomponents}).}}. 


\textbf{Warning} Please note that many Linux distributions come with 
prerelease versions of gcc. These versions are sometimes not as stable
as the normal gcc releases, especially for C++ code.
We discourage using such versions for \textit{STIR} (check with gcc -v).



Other recent C++ compilers should be ok as well, including Intel's compiler
version 11 and CLang.
We would love 
to hear from any attempts of using another compiler. See Section 
\ref{sec:VC}
on how to use Visual Studio's C++ compiler.


Additionally, you need GNU make as well. Other versions of \textit{make} 
will NOT work with our Makefiles.


If you do not have \textbf{gcc} or \textbf{make} you will need to 
install them. Most systems use a package management system to install software. 
If you cannot use that, or you are not the administrator, you 
could follow the instructions given at \R2Lurl{http://gcc.gnu.org/ }{http://gcc.gnu.org/.}



\subsection{
Compilation}

\subsubsection{
Normal Compilation}
\label{sec:normalcompilation}
Use the GNU make utility to compile the program to be run\footnote{{\small Again, 
a non-GNU version of make will not work. It might be that on 
your system GNU make is launched with \texttt{gmake}.}}. The \textbf{Makefile} 
is written for most Unix flavours: we tried it on AIX, Solaris, 
Linux. It can also be used on Windows versions using cygwin (\R2Lurl{http://cygwin.com/}{http://cygwin.com/}). 
Since release 1.2 there is only one \textbf{Makefile}, which is in \textbf{STIR/}.


In addition, the distribution includes a file \textbf{STIR/samples/local/config.mk} 
which you might copy to \textbf{STIR/local/config.mk} and modify according 
to your own needs. See section \ref{sec:local_config.mk}.


The default compilation (BUILD=opt) is optimised and is achieved 
by simply typing:
\cmdline{make all}


It results in the creation of subdirectories in \textbf{STIR/opt} 
with object and executable files (one for each of the subdirectories 
of \textbf{STIR} that contain code).


The 'debug' compilation can be obtained by typing the following 
command:
\cmdline{make BUILD=debug all}
\noindent
This switches off any optimisations and allows the debugging 
of the programs, and results in the creation of subdirectories 
in \textbf{STIR/debug} with object and executable files. Various assert 
statements are enabled in the code, resulting in very slow execution. 
If you experience problems with our software, we recommend running 
the debug version to see if any of these asserts flags the reason.


If you want to copy the executables to a location in your path, 
use
\cmdline{make install INSTALL\_EXE\_DIR=/here}

\noindent
\texttt{INSTALL\_EXE\_DIR} defaults to \texttt{\$(INSTALL\_PREFIX)/bin}, which 
in turns defaults to \textbf{\$\{HOME\}/bin}.


If you want to remove all generated files (such as object files 
and executables), use
\cmdline{make clean}


If you want to remove the installed files, do
\cmdline{make uninstall INSTALL\_EXE\_DIR=/here}

These are only the most basic options. For more targets, 
type
\cmdline{make how}
The following line is useful to check what settings will be used for some
variables (such as \texttt{CFLAGS}). This can be useful to find out
if a particular external library (such as the LLN ecat library) was 
found by the Makefile
\cmdline{make print-make-vars}


One should normally not see warnings when compiling \textit{STIR}
except possibly 'variable not used' type of warnings (although when using
optimization of at least level 2, gcc 3.4 
generates warnings about uninitialised variables in the system library).


The following table gives a summary of all available build options. 
In this table, the notation \$(VAR) means: substitute the value of the
(make) variable VAR.


\newlength{\MakeTableFirstCol}
\newlength{\MakeTableSecondCol}
\setlength{\MakeTableFirstCol}{2.5in}
\setlength{\MakeTableSecondCol}{\textwidth}
\addtolength{\MakeTableSecondCol}{-\MakeTableFirstCol}
\begin{longtable}{|p{\MakeTableFirstCol}|p{\MakeTableSecondCol}|}
\hline
% ROW 1

{\raggedright
\textit{BUILD=build}} & 

{\raggedright
Specifies whether the debugging or optimised version of the 
binaries will be produced. 'BUILD=debug' builds the debugging 
version, 'BUILD=opt' builds the optimised version (default).
'BUILD=nonopt' builds a debugging version without asserts.} \\
\hline
% ROW 2

{\raggedright
\textit{DEST=dir/}} & 

{\raggedright
Put the compilation files in an alternative place. Default is the value
of the \textit{BUILD} make variable (with trailing slash appended).} \\
\hline
% ROW 3

{\raggedright
\textit{WORKSPACE= \linebreak
/absolutedir}} & 

{\raggedright
\textit{Use not recommended} \linebreak
Can be used to tell \textit{make} where the \textit{STIR} subdirectory is, 
but it should find out by itself. The path given should include 
\textit{STIR} as well.} \\
\hline
% ROW 4

{\raggedright
\textit{INSTALL\_EXE\_DIR= \linebreak
/absolutedir}} & 

{\raggedright
Can be used to change the location for executables when using 
'make install'. Defaults to \$(INSTALL\_PREFIX)/bin} \\
\hline
% ROW 5

{\raggedright
\textit{INSTALL\_LIB\_DIR= \linebreak
/absolutedir}} & 

{\raggedright
Can be used to change the location for the library and registry 
object files when using 'make install-all'. Defaults to \$(INSTALL\_PREFIX)/lib} \\
\hline
% ROW 6

{\raggedright
\textit{INSTALL\_INCLUDE\_DIR= \linebreak
/absolutedir}} & 

{\raggedright
Can be used to change the location for the include files when 
using 'make install'. Defaults to \$(INSTALL\_PREFIX)/include} \\
\hline
% ROW 7

{\raggedright
\textit{INSTALL\_PREFIX=/absolutedir}} & 

{\raggedright
See above. It defaults to \$(HOME) which should be your home 
directory. You could set it to /usr/local (if you have write 
permissions there).} \\
\hline
% ROW 8

{\raggedright
\textit{EXTRA\_CFLAGS= \linebreak
'some options'}} & 

{\raggedright
Use to pass extra options to the compiler. For example, if you
use a version of the \textit{boost} files that is not in the standard
include path, you can add the location of its include files to the 
search path of the compiler, e.g. \linebreak
\texttt{EXTRA\_CFLAGS='-I /someloc/boost\_1\_34\_1'}} \\
\hline
% ROW 9

{\raggedright
\textit{EXTRA\_LINKFLAGS= \linebreak
'some options'}} & 

{\raggedright
Use to pass extra options to the compiler when linking} \\
\hline
% ROW 10

{\raggedright
\textit{OPTIM\_CFLAGS, \linebreak
NONOPTIM\_CFLAGS, \linebreak
DEBUG\_CFLAGS}} & 

{\raggedright
Use to pass replace the default optimization options for the compiler when using BUILD=opt, nonopt and debug respectively.} \\
\hline
% ROW 9

{\raggedright
\textit{OPTIM\_LINKFLAGS, \linebreak
NONOPTIM\_LINKFLAGS, \linebreak
DEBUG\_LINKFLAGS}} & 

{\raggedright
Use to pass replace the default optimization options for the linker when using BUILD=opt, nonopt and debug respectively.} \\
\hline
% ROW 10

{\raggedright
\textit{CXX=compiler}} & 

{\raggedright
Use if you want another C++ compiler then g++} \\
\hline
% ROW 11

{\raggedright
\textit{CC=compiler}} & 

{\raggedright
Use if you want another C compiler then gcc} \\
\hline
% ROW 12

{\raggedright
\textit{LINK=compiler}} & 

{\raggedright
Program used for producing the executable, defaults to \$(CXX)} \\
\hline
% ROW 13

{\raggedright
\textit{INSTALL=program}} & 

{\raggedright
This is used by the install targets to copy the files. It defaults 
to cp, but could be set to 'install -m 755 -s' (if you have GNU 
install).} \\
\hline
% ROW 14

{\raggedright
\textit{RM=program}} & 

{\raggedright
This is used by the uninstall targets. It defaults to \texttt{rm -f}.} \\
\hline
% ROW 15

{\raggedright
\textit{DISABLE\_DEPENDENCY\_\linebreak
GENERATION=1}} & 

{\raggedright
By default, the makefile is set-up to generate dependencies 
on the include files. This means that if an include file is changed, 
the relevant .cxx files will be recompiled. This feature is really 
only necessary when you are modifying the \textit{STIR} source files, 
so you could switch it off to saved a little bit of time.} \\
\hline
\end{longtable}


Note that if the \textit{BUILD, DEST, WORKSPACE} options are used at 
compilation time, they also have to be used when using different 
targets (such as \textit{install, clean}).
\subsubsection{
Additional options available for non-parallel code}
\label{sec:compilationgraphics}
The following option can be given to the make command in addition 
to what is specified above

\begin{longtable}{|p{\MakeTableFirstCol}|p{\MakeTableSecondCol}|}
\hline
% ROW 1
{\raggedright \textit{GRAPHICS=value}} & 
{\raggedright Possible values: X, PGM, MATHLINK, NONE. X (default) uses basic 
X windows graphics, PGM writes the graphics to a .pgm file, MATHLINK 
uses the (external) MathLink library which could be used to send 
data directly to Mathematica, NONE switches off all graphics.} \\
\hline
\end{longtable}


If you have problems, you might have to change \textbf{STIR/local/config.mk}
(or in extreme circumstances \textbf{STIR/config.mk}) and introduce options 
specific for your system.


See also section \ref{sec:display}..

\subsubsection{
Additional options available for parallel code}

\textit{STIR} 2.1 contains code for running \texttt{OSMAPOSL} and \texttt{OSSPS} in parallel-mode
\footnote{In fact, any algorithm that uses \texttt{PoissonLogLikelihoodWithLinearModelForMeanAndProjData}.}
(contribution mainly by Tobias Beisel, Univ. of Paderborn). You need a version of the MPI library and
and compiler to get this to work. We have tested this using \R2Lurl{http://www.open-mpi.org/}{OpenMPI} 1.4.2.

Normal build process would be something like\footnote{This will need to fit all on
1 line, or use the continuation character of your shell (normally a backslash).}

\cmdline{make  CC="mpicc.openmpi-gcc" CXX="mpicxx.openmpi-gcc" DEST=mpiopt/ }
\cmdline{\,\,\,\,\,\,\,\,\,\,\,\,\,\,\,iterative/OSMAPOSL iterative/OSSPS}

This assumes that the file \texttt{mpi.h} is the search-path of the compiler. If not,
you could include its location for instance by using the \texttt{EXTRA\_CFLAGS} \textit{make} variable.

\cmdline{make ... EXTRA\_CFLAGS=''-I /where/ever/it/is''}

All the make variables of the previous section apply. Note that at present it is 
recommended that you build this version in a differerent destination directory then
your usual (non-parallel) build. This is because conditional compilation is used
in a few files.

Additional make variables for the parallel version:

\begin{longtable}{|p{\MakeTableFirstCol}|p{\MakeTableSecondCol}|}
\hline
{\raggedright \textit{STIR\_MPI=1}} & 
{\raggedright
Can be used to force compilation for MPI if the CXX compiler is not recognized (see 
\textbf{STIR/config.mk}, which sets it of the compiler name contains the string \texttt{mpi}).
} \\
\hline
{\raggedright \textit{STIR\_MPI\_TIMINGS=1}} & 
{\raggedright
Can be used to enable additional timings for all send/receive pairs 
(see \textbf{distributed\_functions.h} in \textbf{include/stir/recon\_buildblock}. 
This should only be used for testing purposes as it can slow down
the program dramatically.
} \\
\hline
{\raggedright
\textit{MPI\_CFLAGS=...}} & 
{\raggedright
Can be used to change the default MPI compilation flags which currently includes
\texttt{-DMPICH\_IGNORE\_CXX\_SEEK} (see 
\texttt{STIR/config.mk}).
} \\
\hline
\end{longtable}

See section \ref{sec:RunningWithMPI} for information on how to run programs using MPI.

\subsubsection{
Additional options available for threaded code}

Since version 2.0, \textit{STIR} contains preliminary code for running \texttt{FBP2D} in threaded mode
(contribution by Tobias Beisel, Univ. of Paderborn). You need a compiler that 
supports \texttt{OPENMP}. Note however, that this code doesn't presently result 
in a decent speed-up of \texttt{FBP2D}. Normal build process would be something like

\cmdline{make DEST=openmpopt/ analytic/FBP2D STIR\_OPENMP=1}

This assumes that the file \texttt{omp.h} is the search-path of the compiler. If not,
you could include its location for instance by using the \texttt{EXTRA\_CFLAGS} \textit{make} variable.

\cmdline{make ... EXTRA\_CFLAGS=''-I /where/ever/it/is''}


Additional make variables for the threaded version:

\begin{longtable}{|p{\MakeTableFirstCol}|p{\MakeTableSecondCol}|}
\hline

{\raggedright
\textit{OPENMP\_CFLAGS=...}} &
{\raggedright
Can be used to change the default OPEN-MP compilation flags which currently includes
\texttt{-fopenmp} (see 
\texttt{STIR/config.mk}).
} \\
\hline
\end{longtable}

When running a threaded executable, the environment variable \texttt{OMP\_NUM\_THREADS} can be set
to the number of processes to be used for OpenMP. Without using this, the number of threads is 
defined as the number of processors available.

\subsubsection{
Usage of STIR/local/config.mk}
\label{sec:local_config.mk}

Having to type in the above options at every make is quite painful. 
Luckily, the file \textbf{STIR/local/config.mk} comes to your rescue. 
This file is imported (if it exists) by every make (after reading 
STIR/config.mk), and hence can be used to override the defaults 
of all of the above make flags, \textit{except} the \textit{WORKSPACE} flag. 
The distribution contains a sample file \textbf{STIR/samples/local/config.mk}, 
but check it out if it is appropriate for you before copying 
it to \textbf{STIR/local/}.

\subsubsection{
Enabling ECAT 7 support}
\label{sec:ECAT67support}
Older CTI (Siemens) scanners use a file format called ECAT.
At present, \textit{STIR} uses parts of the Louvain la Neuve ecat library (called LLN in
the rest of this document).\footnote{\textit{STIR} versions 1.? came with
specific files for ECAT6 support without the need for the LLN library. 
However, due to license restrictions this is now no longer the case.} 
The library might still be available on  
\R2Lurl{ftp://ftp.topo.ucl.ac.be/pub/ecat }{ftp://ftp.topo.ucl.ac.be/pub/ecat}.
Please check the \textit{STIR}
mailing lists for more information.

You have to download that library and issue 'make --f Makefile.unix' 
(or 'make --f Makefile.cygwin' if you are using CYGWIN on Windows) 
in the new ecat directory first. Please get a recent version 
of this library (dated 20 July 2004 or later) as Merence Sibomana 
has introduced various bug fixes, some of which solve problems 
that you would otherwise experience when using \textit{STIR} and ECAT7 
files.


To compile and link ifheaders\_for\_ecat7.cxx and the other files 
that enable ECAT support, \textit{STIR} needs to know the location 
of the LLN files. The following 2 \textit{make} options allow you 
to set this:



\begin{longtable}{|p{\MakeTableFirstCol}|p{\MakeTableSecondCol}|}
\hline
% ROW 1
\raggedright
\textit{LLN\_INCLUDE\_DIR=\linebreak
/loc/of/includes} &
This needs to point to the path of the LLN include files. Default 
value is \textit{\$(WORKSPACE)/../lln/ecat}, which by default expands 
to \textbf{STIR/../lln/ecat}\\
\hline
% ROW 2
{\raggedright \textit{LLN\_LIB\_DIR= \linebreak
/loc/of/lib}} & 
{\raggedright This needs to point to the directory where libecat.a is produced. 
Default value for \textit{LLN\_LIB\_DIR} is \textit{\$(LLN\_INCLUDE\_DIR).}} \\
\hline
\end{longtable}


Once these options are set correctly (for example in STIR/local/config.mk), 
ECAT support will be automatically enabled after issuing \textit{make}. 
If the file ifheaders\_for\_ecat7 is not compiled, it probably 
means that the above flags point to an incorrect location.

\subsubsection{
Enabling AVW support}
\label{sec:AVWsupport}
If you have the \textit{AVW}\texttrademark{}  library
\footnote{See \R2Lurl{http://www.mayo.edu/bir/Software/AVW/AVW1.html}
{www.mayo.edu/bir/Software/AVW/AVW1.html}.
} installed on your system, the make process should automatically built
some \textit{STIR} routines to be to use it for IO (see \ref{sec:convAVW}).

To compile and link \textit{AVW} support, \textit{STIR} needs to know the location 
of the \textit{AVW} files. 
They are set in \texttt{config.mk} according to recommend \textit{AVW} values,
i.e. using the environment variables \texttt{AVW} and \texttt{TARGET}.
The following \textit{make} options allow you 
to set this:

\begin{longtable}{|p{\MakeTableFirstCol}|p{\MakeTableSecondCol}|}
\hline
% ROW 1
\raggedright
\textit{AVW\_INCLUDE\_DIR=\linebreak
/loc/of/includes} &
This needs to point to the path of the AVW include files. Default 
value is \textit{\$(AVW)/include}\\
\hline
% ROW 2
{\raggedright \textit{AVW\_LIBS= \linebreak
/loc/of/lib}} & 
{\raggedright This needs to point to the AVW library.
Default value is \textit{\$(AVW)/\$(TARGET)/lib/libAVW.so}.
Note that this default is inappropriate on Windows.} \\
\hline
% ROW 3
{\raggedright \textit{HAVE\_AVW= \linebreak
0 or 1}} & 
Can be used to remove AVW support even if the library is found.\\
\hline
\end{longtable}
Note that the preprocessor symbol \texttt{HAVE\_AVW} will be defined 
if all files have been found (see \texttt{config.mk}).


Once these options are set correctly (for example in \texttt{local/config.mk}), 
AVW support will be automatically enabled after issuing \textit{make}. 
If the file \texttt{conv\_AVW} is not compiled, it probably 
means that the above flags point to incorrect locations.


\subsubsection{
Compilation using Visual C++}
\label{sec:VC}

Old versions of \textit{STIR} could be compiled with Visual C++ 6.0, service pack 5 but there are no recent 
reports if this version still works. We very much recommend to upgrade your compiler. The Visual
Studio Express versions are free. Version 7.1 (i.e. 2003) might still be fine. \textit{STIR} 2.1
should normally compile with Visual Studio 2008 and 2010.

There are two different ways to build using Visual C++ (but see also \ref{sec:UsingCMake}
which provides a much easier way to build). Both 
are somewhat fiddly. Before we go there, we have to got a problem out-of-the-way.

{ \subsubsubsection{CPUTimer problems}
}
At present, \textit{STIR} uses some files available in the Windows API for 
getting accurate CPU time estimates. If you do not have the Windows API
installed, you might get compilation errors mentioning that 
the file \texttt{windows.h} cannot be found. A work-around is to
define the preprocessor symbol \texttt{STIR\_CPUTimer\_use\_clock}. 
This does mean that \textbf{CPUTimer} will return wall-clock time.
If you are using \textbf{make}, you can put a line
\begin{verbatim}
  EXTRA_CFLAGS=-DSTIR_CPUTimer_use_clock
\end{verbatim}
in your \textbf{STIR/local/config.mk}, see section \ref{sec:local_config.mk}.
If you are using the Visual Studio GUI, you can either add this preprocessor
variable to all projects, or you can put a line
\begin{verbatim}
  #define STIR_CPUTimer_use_clock
\end{verbatim}
at the start of \texttt{include/stir/CPUTimer.inl}.


{ \subsubsubsection{Using make}
}


If you have a Unix compatible \textbf{make} program and possibly some 
other Unix-like programs such as \textbf{sh}, then life could be as 
simple as typing
\cmdline{make CXX=cl}


Kris Thielemans has tested this with the make program included in the CYGWIN 
distribution (\R2Lurl{http://cygwin.com }{http://cygwin.com}). 



See section \ref{sec:normalcompilation} et al. for general info on the make parameters. 
Specific information to VC follows now.



You have to make sure that the location of the Visual C++ files 
is known. In particular, you will need to set the path and the 
environment variables \textbf{LIB} and \textbf{INCLUDE}. This is normally 
done by running the file \textbf{vcvars32.bat} supplied with Visual 
C++. Note that when using CYGWIN, the easiest way to do this 
is to include a line 
\begin{verbatim}
call vcvars32.bat
\end{verbatim} 
in your  \textbf{cygwin.bat}.



Where we rely on external programs such as \textbf{cp}, it is generally 
possibly to either replace these by setting a \textit{make} variable, 
or by switching the relevant lines off. Do some investigation, 
and let the mailing list know. So, if you do not have cygwin, 
a possibly more realistic line is
\cmdline{make CXX=cl DISABLE\_DEPENDENCY\_GENERATION=1 {\textbackslash}\\
 INSTALL=copy 
RM=del GRAPHICS=PGM install}


Getting ECAT support is even more of an adventure, as it requires 
some changes to the Makefile supplied with the LLN Matrix library. 
See the mailing lists.

{ \subsubsubsection{Using the Visual Studio interface}
}
In addition to the source files, you will need to download VCprojects.zip. 
Extract it in the same place as you extracted the source files. 

\textbf{Currently there is a linking problem with some of the utilities using
the distribution solution files} (tested using VC 2010). A workaround would be to move all files
in the \textbf{Shape\_buildblock} and \textbf{numeric\_buildblock} projects to the \textbf{buildblock}
project.

Note that since \textit{STIR} 1.3, the Visual Studio projects might be 
out of date. You essentially have to add all source files in the
\textit{STIR} sub-directories to a relevant project (e.g., add 
all .cxx files in \texttt{buildblock} directory to the \texttt{buildblock}
project.

You will find several solutions in the \textbf{STIR{\textbackslash}VC} subdirectory. 
In particular, \textit{reconstructors.sln} contains all reconstruction 
programs.



\textbf{Note:} The project files enable ECAT support by default. 
If you do not want ECAT support (i.e. if you do not (want to) 
have the LLN ecat library), you probably have to remove the \textit{HAVE\_LLN\_MATRIX} preprocessor 
define, and remove dependencies on the ecat project. This might 
all be avoidable, but we didn't do this for you. Please feel 
free to find a neat way around all this ugly stuff and let us 
know.

\textbf{Note:} The projects and workspaces in the distribution contain references to
files and projects called \texttt{local*}. You will probably have to delete those
manually from the projects.


At present, \textit{STIR} uses parts of the Louvain la Neuve ecat library, see \ref{sec:ECAT67support}. 
The following are instructions specific to using VC projects.

Extract 
the library in the following subdirectory of your home directory: \textbf{lln{\textbackslash}ecat}.\footnote{{\small You 
no longer have to issue 'nmake /f Makefile.win32' in the ecat 
directory first. The ecat.dsp project takes care of that. Using 
the project has the advantage that you don't get error messages 
anymore when trying to build a debug version.}} In addition, get 
the \textbf{ecat.dsp} file from the same source and put it in a new 
subdirectory \textbf{lln{\textbackslash}ecat{\textbackslash}VC}.



When opening the workpackages the first time, and if you want
ECAT 6 and 7 support, you will need to tell Visual 
Studio the location of the \textbf{ecat.dsp} project file (which should 
be in the VC subdirectory of the ecat distribution).


There are various graphical display modes supported. To set them:\\
1.\tab 
Choose Settings in the Project menu\\
2.\tab 
Select all projects you want to modify (for the libraries, this 
is only necessary for the \textit{display} project)\\
3.\tab 
Choose C/C++ tab and go to Preprocessor options\\
4.\tab 
Add a predefined preprocessor value appropriate to the display 
mode you want. Supported values are: 


STIR\_SIMPLE\_BITMAPS uses X Windows (simple display, no menus).


STIR\_MATHLINK sends the image information over a MathLink connection 
to Mathematica (or any other MathLink program) where it can be 
displayed anyway you like.


STIR\_PGM puts all images in a single PGM file.


Multiple preprocessor symbols may be defined. The distributed 
version uses STIR\_PGM.


If your mode selection requires an additional library (STIR\_MATHLINK 
needs the ml32iw.lib library distributed with Mathematica, STIR\_SIMPLE\_BITMAPS 
needs X windows libraries which you would have to get from somewhere 
else), set the environment variable to PARAPETGRAPHLIB to the 
name of the library you want. Note that this has to be done BEFORE 
you start Visual Studio. In Windows NT, 2000 or XP or later, environment 
variables can be set using the System Control Panel. On Windows 
9?, you have to modify autoexec.bat. 



When creating new projects, make sure that RTTI is enabled (Settings, 
C/C++ tab, General). 



Some more info on using these projects might be available in 
the archive of the stir\_users mailing list.

\subsubsection{
Experimental feature: using CMake}
\label{sec:UsingCMake}
Since version 2.1, \textit{STIR} contains files to build using the platform-independent
\R2Lurl{"http://www.cmake.org"}{CMake}, based on example files provided by 
Robert Barnett (SWAHS, Sydney). This is much better for configuration and finding 
system dependencies.
However, tests are not yet included in the CMakeFiles.

See the CMake web-site more information, and in particular\\
\R2Lurl{http://www.cmake.org/cmake/help/runningcmake.html}{http://www.cmake.org/cmake/help/runningcmake.html}.

Normal build procedure on Linux/Unix would be something like

\cmdline{cd /somewhere/nice/STIRbuild/Release}
\cmdline{ccmake /somewhere/else/where/the/source/is/STIR}

This will bring up a basic interface that allows you to specify build variables. For example,
if CMake cannot find \textit{boost}, you will have to set \texttt{BOOST\_ROOT}. You can also
specify locations for the LLN files, AVW libraries, wether to use \textit{OPENMP}, or
\textit{MPI}, set \textit{GRAPHICS} according to what it can find on your system, etc. 
See the section \ref{sec:normalcompilation} for explanation of some
of the variables. Note that you can use e.g. \textit{DISABLE\_LLN\_MATRIX} to \textit{not}
use the LLN ECAT library, even if cmake found it. Finally, do not forget to set
\texttt{CMAKE\_BUILD\_TYPE} to \texttt{Release}, \texttt{Debug} or \texttt{RelWithDebug}.

You can press \texttt{c} to do additional configurations, \texttt{g} for generating
your build files, \texttt{q} to quit. After that, you will have to use your normal
build system (i.e. \texttt{make} on Linux/Unix).

On Windows, the procedure is essentially the same, except that you have to specify locations 
via the cmake GUI. You can also select your build system (e.g. Visual Studio 2008). In that case,
you do not need to specify the Build Type, as you will be able to select that from in 
Visual Studio.



\subsubsection{
Operating system specifics}

{ \subsubsubsection{Mac OS}
The lln library does not compile on Mac OS. 
}

{ \subsubsubsection{All Unix/Linux flavours}
}

If you want to use the X windows display routines, the Makefile 
supplied should work out-of-the-box if you have suitable libraries installed. 
You can diagnose this as follows.

{ \subsubsubsubsection{X development libraries}
}

If you experience compilation 
or linking problems mentioning X11, you have to check if the 
X development libraries are installed on your system. You can 
check this by doing
\cmdline{find /usr -name Xlib.h -print}

If this file is not found, you'll have to install these libraries 
somehow (on Linux you would normally use \texttt{apt-get} or \texttt{rpm}).
If after this, Xlib.h is not found while compiling, 
either set the environment variable C\_INCLUDE\_PATH (when using 
gcc) or use \\
\texttt{EXTRA\_CFLAGS='-I /usr/X11R6/include' }\\
or whatever suitable path (the quotes are necessary when setting 
EXTRA\_CFLAGS on the command line, not when setting it in 
STIR/local/config.mk).\\
If while linking the libX11.a library is not found, it is not 
in the default path of your linker. First find out where it is:
\cmdline{find /usr -name libX11.a --print}

Then either set the environment variable LIBRARY\_PATH or use \\
\texttt{EXTRA\_LINKFLAGS='-L /usr/X11R6/lib/'}\\
 or a suitable path.

{ \subsubsubsubsection{curses library}
}

You also need to install the curses or ncurses package (for some reason, KT experienced 
a linking error when trying to use the curses library). As mentioned above, on Ubuntu and
Debian, you can do

\cmdline{apt-get install libncurses5 libncurses5-dev}

prefixed with \texttt{sudo} if you are not logged in as root. For other Linux versions, 
you should be able to find similar RPM packages online.

{ \subsubsubsubsection{X display depth}
}

\textit{STIR} version 1.0 required that your Xserver worked in 8bit 
mode (or Pseudocolor mode in X terminology). This is no longer 
necessary. However, if you are experiencing problems with the 
display, you could try 8bit mode anyway.



{ \subsubsubsection{Cygwin on Windows}
}

Cygwin is a must-have if you are using Windows but would like 
to have nearly everything that Linux/Unix has to offer (at least 
from a user's point of view). Check out http://cygwin.com.

{ \subsubsubsubsection{Using X windows on cygwin}
}
\begin{itemize}
\item Install the Xorg-x11 packages using the cygwin setup utility 
(you need the devel package).
\item 
install ncurses-devel if you don't have it yet (use the cygwin 
Net setup).
\item make sure that /usr/X11R6/bin is in your PATH (necessary 
for DLLs)
\item
make sure gcc will find the X11 files. This should happen automatically, 
but if not you either set the 
environment variables C\_INCLUDE\_PATH and LIBRARY\_PATH, or 
do
\cmdline{make GRAPHICS=X EXTRA\_CFLAGS='-I /usr/X11R6/include' 
{\textbackslash} \\
EXTRA\_LINKFLAGS='-L 
/usr/X11R6/lib/'}
\end{itemize}

You can start the Xorg X server by executing the startxwin.sh 
script (you might want to modify this a bit to suit your taste). 



Other decent X servers should work as well (Kris Thielemans used Exceed at 
some point).


\subsubsection{
Running tests}

We \textbf{highly recommend} running test programs after building. Change 
to your \textbf{STIR} directory, and type
\cmdline{make run\_tests}


This will compile all test programs, and run those which do not 
require user input. You should see a sequence of messages, all 
ending with \texttt{All tests ok}. (The make process will fail 
when any of the tests fails.). You may also wish to run \texttt{make 
run\_interactive\_tests}. It is recommended to do these tests 
also with the version compiled in debug mode (as that implies 
more tests). 
\cmdline{make run\_tests BUILD=debug}


If everything is all right, you can free up some disk space by 
using
\cmdline{make clean\_tests}
\cmdline{make clean\_tests BUILD=debug}

We also provide \textbf{STIR/test/run\_tests.bat} for running the automatic 
tests on Windows when using Visual Studio and the VCprojects.zip 
files. This will currently not run any tests in \textbf{STIR/recon\_test} though.


On our Web page, we also provide test reconstructions as part of the
\textbf{recon\_test\_pack}, with an 
automated procedure to see if you can reproduce the correct results. 
You really should download this and run it.


Also, if you are using a non-standard scanner, 
you might want to run one reconstruction on your data with 
the debug version of the reconstruction program, to see if everything 
works as intended. You'll probably want to choose a small value 
for the `maximum absolute segment number to process' parameter 
in your reconstruction (see Section \ref{sec:OSMAPOSL}), as this will be \textbf{\textit{very}} 
slow.


\section{
Running \textit{STIR} programs}

Here we describe:\\
{\textbullet} documentation and program conventions\\
{\textbullet} supported file formats \\
{\textbullet} rebinning algorithms\\
{\textbullet} reconstruction programs\\
{\textbullet} utility programs\\
{\textbullet} user-selectable components\\
{\textbullet} display properties


\subsection{
Conventions}
\label{sec:conventions}
When discussing command line parameters of the \textit{STIR} executables, 
the following format is used in this documentation (and the usage messages):
\cmdline{executable\_name parameter1 parameter2 {\textbackslash}\\
{[}optional\_parameter3 [optional\_parameter4 etc]]}


This means that the first 2 parameters are mandatory, but a 3$^{rd}$ 
parameter can be given, or 4 parameters. A single parameter is 
usually given as one word, but sometimes as a string between 
\texttt{<} \texttt{>}. 


All parameters have to be on a single command line. This is indicated 
by the backslash {\textbackslash}, as in Unix this is the standard way 
of continuing a command line on the next line.


All executables are supposed to be located in the path of the 
command shell used.


Most \textit{STIR} programs accept a single parameter on the command 
line, which is usually optional

\cmdline{executable\_name [parameter\_filename]}


The parameter file is a text file which uses an Interfile-like 
syntax. It is composed of keywords, corresponding to the names 
of the various parameters, with the values entered next to them. 
Spaces and tabs are normally irrelevant. Parameters omitted from 
the parameter file are assigned a default value. 


If a parameter file is not passed to the executable, the user 
is prompted for the required information. For questions that 
ask for a number, the format is as follows:


\tab What number do you want to enter today \\
\tab [minimum, maximum, D: default]:


If a simple Carriage Return is entered, the default value is 
selected.


Sample parameter files for particular programs (or part of a 
parameter file relevant to a particular common component) are 
presented in subsequent sections. See also the files in \textbf{STIR}/\textbf{samples}.



All \textit{STIR} executables returns a status value to signify success 
or failure. What value this is depends on your Operating System. 
On Unix, Linux and Windows variations, success is indicated by 
a status of 0, and failure by anything else. There is currently 
no differentiation between the reasons for failure.

\subsection{
Running programs using MPI \label{sec:RunningWithMPI}}
If you have compiled with STIR\_MPI, you need to the executables in such a way that
they know about the availables nodes. How to do this is system/MPI version specific. The following
should work with OpenMPI on Linux:

\cmdline{mpirun.openmpi -np 12 --hostfile ~/myconf.txt OSMAPOSL mypars.par}

\noindent 
where the host file describes your set-up, e.g. if you have 3 nodes with differing number of cores, the
host file could loook like this:
\begin{verbatim}
# The Hostfile for Open MPI
beo-09 slots=4
beo-10 slots=8
beo-11 slots=8
\end{verbatim}
Without host file, the executables will normally be run on the host where you executed the command, which
is useful for multi-processor machines. 

\textit{MPICH} would use a similar line using \texttt{mpiexec}.

If you have a queueing system that supports MPI such as 
\R2Lurl{http://www.clusterresources.com/products/torque-resource-manager.php}{Torque}, it is better to use 
that system for sorting out available resources etc. For example, to use 2 nodes with 4 processes each,
the following should work

\cmdline{qsub -I -l nodes=2:ppn=4}

This will open a interactive prompt on one node where you just type

\cmdline{mpirun.openmpi OSMAPOSL mypars.par}

Of course, you can use \texttt{qsub} to submit a job as opposed to getting an interactive prompt.

\subsection{
File formats}

The STIR utility and reconstruction programs frequently need 
to read and write files of image and projection data. Files formats 
are encountered in which data and header information are maintained 
in separate files (e.g. interfile). In other formats, data files 
carry header information (e.g. the native GE Advance sinogram 
format).

\subsubsection{Interfile}

The most comprehensively supported file format in the library 
is a newly proposed version of interfile. More details about 
this type can be found in the ''Other info'' section of the \textit{STIR} website.

It is currently the only format supported for writing projection 
data (except by the \textit{conv\_to\_ecat?} utilities\textit{)}. Projection 
data files are written in pairs:\\
\textit{projdata\_filename.hs, 
projdata\_filename.s}\\
where \textit{projdata\_filename.hs} is the header text file and \textit{projdata\_filename.s} 
is the data file. Please read section \ref{sec:outputinterfile} for info regarding 
images written by \textit{STIR} in Interfile.

\subsubsection{VOLPET sinograms}

The GE VOLPET sinogram format is supported for reading, but for
the Advance only, 
however data files must have a single data set, maximum ring 
difference 11 and contain 281 bins * 256 segments * 336 views. 
We do have software that read VOLPET headers for all GE scanners, but can currently
not distribute it due to licensing restrictions. If you have the
GE PET Toolbox package you can ask Kris Thielemans for VOLPET and RDF
support in \textit{STIR}.

\subsubsection{ECAT6 and ECAT7 data} 
See section \ref{sec:ECAT67support} for enabling support for ECAT data.

ECAT6 images can be read without conversion. However, only the 
first frame (ECAT matrix 1,1,1,0,0) will be read.

ECAT7 sinograms, attenuation files and images can be read without 
conversion. However, only the first frame (ECAT matrix 1,1,1,0,0) 
will be read.
\textbf{Warning} The calibration factor field in the main header of ECAT7 images
is currently ignored.

For dynamic or gated data, conversion can be used, see \ref{sec:convertingdata}.

\subsubsection{Image IO using the AVW library}
First steps are made to be able to use the \textit{AVW} library. See the
\texttt{conv\_AVW} utility (section \ref{sec:convAVW}).



See section \ref{sec:outputfileformats} for supported output file formats. In addition, 
utilities are available for converting ECAT6 and ECAT7 to/from 
interfile (see Section \ref{sec:convertingdata}). 
These should be used for dynamic 
or gated scans.

\subsubsection{SimSET files}
There are some preliminary files for make it easier to use SimSET and \textit{STIR} together in the
\textbf{SimSET} directory. See the README.txt in that directory for more info.

\subsection{
List mode processing}

Some scanners produce list mode data, which is essentially a 
list of events.  \textit{STIR} provides utilities to use
the list mode files, for example to convert them to sinograms. It is
also possible to reconstruct images directly from list mode data, although
this has not been tested very well yet as of \textit{STIR 2.1}.

Currently supported list mode formats are specific to the ECAT 
HR+ and ECAT EXACT 3D scanners. There are some unfinished classes
available on the \textit{STIR} web-site to read \textit{LMF} format files,
in conjunction with the \textit{LMF} library. However, these are obsolete
as the OpenGATE project distributes files
to enable STIR to read \textit{LMF} format files.

\subsubsection{
lm\_to\_projdata}

This utility can be used to bin (or sort) the list mode data 
into 'projection data' (also known as \textbf{3D sinograms}. 
This can then be processed further by other \textit{STIR} utilities. It 
needs to be run as follows:

\cmdline{lm\_to\_projdata par\_filename}


See \textbf{STIR/samples/lm\_to\_projdata.par} for an example file. Please 
check out the online documentation (as generated by doxygen) 
for more info.

\subsubsection{
lm\_to\_projdata\_bootstrap}
As before, but using bootstrapping (with replication). This is useful to generate multiple realisations 
from a single list mode file to study variance.

You can use it as follows:

\cmdline{lm\_to\_projdata\_bootstrap par\_filename seed}

The seed is a positive integer (do not use $0$). Specifying different seeds will give you different noise realisations.
You can use the same parameter file as for \texttt{lm\_to\_projdata}.

\subsubsection{
lm\_fansums}
Computes fan-sums, i.e. for each detector, sum all data that are in coincidence with that detector.
Results are written as a simple text file. This could be used to check if all detectors are working
properly for instance.

\subsubsection{
list\_lm\_events}
Allows inspecting the events in a list mode file.

\subsection{
Rebinning algorithms}

In 3D PET, the name \textit{rebinning} is used for the 
process of manipulating the 3D projection data set to the equivalent 
of 2D projection data. This reduces the number of segments (see 
\textit{STIR} Glossary) to 1. Popular rebinning algorithms are SSRB, FORE 
and variations such as FOREX, and FOREJ. Rebinning is often used 
before reconstruction to decrease total reconstruction time.


\subsubsection{
SSRB}
\label{sec:SSRB}
The \textit{Single Slice Rebinning} algorithm [Dau87] is the oldest 
and simplest rebinning algorithm. It essentially ignores the 
obliqueness of a Line of Response and moves data to the axial 
position in segment 0 such that z-resolution on the axis of the 
scanner is preserved.


The \textit{STIR} implementation of SSRB is a generalisation that applies 
the same idea while still allowing preserving some of the obliqueness. 
For instance, for a dataset with 9 segments, \textbf{SSRB} can produce 
a new dataset with only 3 segments. This essentially increases 
the axial compression (or \textit{span} in CTI terminology), see the 
\textit{STIR} Glossary on axial compression. In addition, \textbf{SSRB} can 
introduce extra \textit{mashing} (see the \textit{STIR} Glossary) of the data, 
i.e. add views together.


\textit{Usage:}

\cmdline{SSRB output\_filename input\_projdata\_name {\textbackslash} \\
num\_segments\_to\_combine [num\_views\_to\_combine {\textbackslash}\\
{[}do\_normalisation [max\_in\_segment\_num\_to\_process ]]]}

\begin{description}
\item[num\_segments\_to\_combine] has to be odd. It is used as the 
number of segments in the original data to combine.
\item[num\_views\_to\_combine] has to be at least 1 (which 
is the default). It is used as the number of views in the original 
data to combine.
\item[do\_normalisation] has to be 1 (default) or 0. When it is 1, 
the result is normalised, i.e. divided by \textit{num\_segments\_to\_combine*num\_views\_to\_combine}. 
This is appropriate for rebinning data where normalisation has 
already been applied, but inappropriate otherwise.

\item[max\_in\_segment\_num\_to\_process] defaults to all segments. 
Can be used to ignore the most oblique segments in the input 
data. Note that the most oblique segments that cannot be rebinned 
completely will be ignored automatically. For instance, when 
the input data has 7 segments, and \textit{num\_segments\_to\_combine} is 
3, only 3 segments are used from the input (and 1 output segment 
produced), as 9 input segments would be necessary to produce 
3 'complete' output segments.
\end{description}


\subsection{
Image reconstruction programs}




\subsubsection{
Iterative algorithms}

Two statistical reconstruction algorithms are currently implemented:
\begin{description}
\item[OSMAPOSL] an implementation 
of the OSEM-One Step Late algorithm with various additional refinements 
and capabilities. See [Jac00] for a description of many details of the
implementation.
\item[OSSPS] an implementation of the Ordered Subsets Paraboloidal
Surrogate algorithm by Erdogan and Fessler, and Ahn and Fessler [Ahn2003]. 
\end{description}

Since \textit{STIR} 2.0, all iterative algorithms use a \textit{generalised objective 
function}\footnote{We call them \textit{generalised} because they do not
necessarily have to correspond to a function. Most algorithms only require
that we can compute some kind of gradient. As an example, the update
of OSEM with the Median Root Prior (MRP) does not correspond to the
gradient of a function.} which the algorithm tries to maximise. The result
of the algorithm will obviously depends very much on the objective function
used.

Currently, only Poisson-based objective functions are implemented in \textit{STIR}. We
have an objective function for static projection data (\textit{aka} sinograms),
list mode data and parametric images from dynamic projection data 
(currently only using the Patlak model).

Most parameters are shared between algorithms and objective functions. Therefore,
once you understand how to run e.g. OSEM, it should be easy to understand OSSPS etc.
This guide therefore describes the case of running OSMAPOSL on projection data in detail
first. Other algorithms (and objective functions) are then described as variations on this
case.

{ \subsubsubsection{Running OSMAPOSL on static projection data}
}
\label{sec:OSMAPOSL}

The program \textbf{OSMAPOSL} executes the IF-OSEM-OSL algorithm. 
This is a generalisation of the One Step Late algorithm [Gre90]. In 
particular, the prior can be set to the Median Root Prior [Ale97]. \textbf{OSMAPOSL} 
allows \\
-\tab 
the use of subsets, similar to the OSEM algorithm [Hud94]. OSMAPOSL
will only work when the subsets are approximatelly balanced (i.e.
the number of subsets has to be a divisor of the number of views).
In fact, by default this implementation\footnote{In fact, this assumption
is part of the \texttt{PoissonLogLikelihoodWithLinearModel} class, which is
the base class of the objective function that OSMAPOSL uses.} assumes that the 
subsensitivity image is proportional to the total sensitivity image,
but this can be switched off .\\
-\tab 
inter-update filtering (applying a filter to the image before 
it is multiplied with the update image) [Jac00]\\
-\tab 
inter-iteration filtering (applying a filter after the rest of 
the image update is performed), sometimes called EMS [Sil90], 
see also [Mus04].\\
-\tab 
post-filtering (apply a filter after the last subiteration)\\
-\tab 
applying the prior information in an additive or multiplicative 
way [Mus01]\\
-\tab 
random permutation of the order of the subsets in each iteration


See [Jac00] for more information, and also the online documentation 
for the class \textbf{OSMAPOSLReconstruction.}


The successive image iterates of the algorithm are saved at pre-specified 
subiteration intervals in a sequence of image files. The files 
are named with a pre-specified output file prefix with the subiteration 
number appended after an underscore.


As described in Section \ref{sec:conventions}, 
the recommended manner of running \textbf{OSMAPOSL} 
is to pass the executable a parameter file argument which specifies 
the relevant parameters of the reconstruction. Parameters omitted 
from the file are assigned a default value. If a parameter file 
is not used, the program prompts the user for the required information. 



The form of a typical parameter file\footnote{Parameter values used here are for
illustration only and are \textit{not} recommended.} is as follows:

\begin{verbatim}
OSMAPOSLParameters :=
;lines starting with semicolons are comments
objective function type:= \
  PoissonLogLikelihoodWithLinearModelForMeanAndProjData
PoissonLogLikelihoodWithLinearModelForMeanAndProjData Parameters:=
; input, sensitivity and prior parameters here

input file := projection_data_filename.hs

; use -1 to use the maximum available
maximum absolute segment number to process := 4
zero end planes of segment 0 := 1

; keywords that specify the projectors to be used
Projector pair type := Matrix 
 Projector Pair Using Matrix Parameters := 
   Matrix type := Ray Tracing
     Ray Tracing Matrix Parameters:=
   End Ray Tracing Matrix Parameters:= 
 End Projector Pair Using Matrix Parameters :=

; background (e.g. randoms)
additive sinogram := 0

; sensitivity related keywords
 ; time frame info used for dead-time calculation when using ECAT7
 ;time frame definition filename:=
 ;time frame number:= 1

 ; normalisation and attenuation info
 ; Bin Normalisation type:= None
 
 recompute sensitivity := 1
 use subset sensitivities:=1 ; recommended (but default is currently 0)
 ; optional filename to store/read the sensitivity image 
 ; (if use subset sensitivity is off)
 ;sensitivity filename:=
 ; optional filename to store/read the subsensitivities
 ; use %d where you want the subset-number (a la printf)
 subset sensitivity filenames:= sens_%d.hv

; keywords for specifying the prior information
prior type := None

; next keywords can be used to specify image size, but will be removed
; they are ignored when using an initial image
zoom := 1
; use --1 for default sizes that cover the whole field of view
XY output image size (in pixels) := -1
end PoissonLogLikelihoodWithLinearModelForMeanAndProjData Parameters:=


; set output file format, if omitted a default value will be used
Output file format := Interfile 
Interfile Output File Format Parameters := 
; byte order := little-endian 
; number format := signed integer 
; number of bytes per pixel := 2 
End Interfile Output File Format Parameters :=

initial estimate:= initial_image_filename.hv
enforce initial positivity condition:=1
number of subsets:= 6
start at subset:= 0
number of subiterations:= 30
start at subiteration number:=1

output filename prefix := out_file
save estimates at subiteration intervals:= 2
uniformly randomise subset order:= 1


; keywords that specify the filtering that occurs after every subiteration
inter-iteration filter subiteration interval := 4
inter-iteration filter type := Separable Cartesian Metz
 ; keywords below will depend on the filter type (see text)
 separable cartesian metz filter parameters := 
   x-dir filter fwhm (in mm) := 6 
   y-dir filter fwhm (in mm) := 6 
   z-dir filter fwhm (in mm) := 6 
   x-dir filter metz power := 2 
   y-dir filter metz power := 2 
   z-dir filter metz power := 2 
 end separable cartesian metz filter parameters := 


; keywords that specify the filtering that occurs at the end
; of the reconstruction
post-filter type := None

; keywords that specify the filtering that occurs before 
; multiplying with the update image

inter-update filter subiteration interval := 4
inter-update filter type := None

map model := additive

; keywords for preventing too drastic (multiplicative) updates
; below just set to their default values
maximum relative change := 3.40282e+38
minimum relative change := 0

; enabling this will write the multiplicative update images 
; every sub-iteration
write update image := 0

END :=
\end{verbatim}

See \textbf{STIR/samples/} for example parameter files.

The following gives a brief explanation of the parameters. Where 
appropriate, the notation \textit{[min, max, default]} is used to 
specify allowable numeric ranges. Values of \textit{\ensuremath{\infty}} indicate 
that the largest value possible for the numeric type of the parameter 
value is allowed.

{ \subsubsubsubsection{Objective function related parameters (Poisson statistics)}
}
\label{sec:PoissonProjectionDataObjectiveFunction}

\begin{description}

\item[input file]
The name of the file for the measured projection data. 


\item[zero end planes of segment 0]
If this is set to 1, the reconstruction pretends that the projection 
data measured in the extreme-most rings of the scanner (the end 
planes of segment 0) are zero. See also the discussion in Section 
\ref{sec:OSMAPOSLtechnotes}.


\item[time frame definition filename]
See \textit{Bin Normalisation type}

\item[time frame number]
Defaults to 1. Currently only used for finding the frame number and hence dead-time (in ECAT7).
See \textit{Bin Normalisation type}.

\item[Bin Normalisation type]
This keyword is used to set both normalisation factors and attenuation
correction factors. See section \ref{sec:binnormalisation}.

\item[use subset sensitivities]
Defaults to 0 but it is recommended to set it to 1.
By default, it is assumed that all subset sensitivities are equal, and therefore proportional 
to the total sensitivity. We can then save some memory by internally storing only the total sensitivity. 
However, if you have non-uniform attenuation, this assumption will generate some artefacts in the image.

\item[recompute sensitivity]
Defaults to 0. If \textit{sensitivity filename} or \textit{subset sensitivity filenames} 
are specified and recomputation is switched off, the sensitivity data will be read
from file, and the 
\textit{Bin Normalisation type} keyword will be effectively ignored.
On the other hand, if the filename is set and recomputation is switched on, the
(subset) sensitivity will be written to that filename. This parameter might be useful
to avoid recomputing the sensitivity if it does not change.

\item[sensitivity filename]
The sensitivity image is an important ingredient of 
likelihood-based reconstruction algorithms of PET or SPET emission data
\footnote{In \textit{STIR} 2.0, this is generalised to the sensitivity for
any problem with Poisson statistics where the data are linear
combinations of the variables of interest.}
For each voxel, it represents the total probability of detection of an 
event originating in that voxel (up to a global proportionality factor).
It is computed by backprojecting projection data with all elements set to 1.

The name of the file containing the sensitivity image. The 
default is to compute it from the previous keywords. You could 
set it to 1, which is am image uniformly 1 over all voxels. 
Using this is generally a bad idea however. It is 
only somewhat appropriate when using fully precorrected data 
in 2D PET. Even in that case, you will get an image with the 
wrong scale factor.

Note that when no attenuation nor normalisation was used, 
the sensitivity will be computed based on geometrical backprojection
only. This only appropriate when reconstructing 
from pre-corrected projection data.

\item[subset sensitivity filenames]
This should be used instead of \textit{sensitivity filename} when using subset
sensitivities. It should provide a pattern \'{a} la \texttt{printf}\footnote{In fact,
\texttt{boost::format} is used, so the pattern can be more flexible.} to allow
constructing different filenames for each subset.


\item[Projector pair type]
Specifies the back/forward projector pair to be used in the reconstruction. 
See Section \ref{sec:projectorpairs} for possible values. We recommend using matching
projectors (\textit{e.g.} based on a single projection matrix).


\item[additive\_sinogram]
This parameter can be used to take randoms or scatter into account. 
It is a constant background term that will be added to the forward 
projected data in the denominator of the MAP-OSL algorithm. Please 
note that currently the forward projector produces 'geometric' 
data. This is because the normalisation and attenuation coefficients 
can be cancelled out between the forward and backprojector. They 
then remain only in the sensitivity image and the background 
term. This means that in effect, a normalised and attenuation 
corrected randoms sinogram would have to be passed as 'additive 
sinogram'. By suitably adding a term to the input projection 
data, the Shifted Poisson version of OSL can be run as well.


\item[maximum absolute segment number to process]
Denoting the input value by \textit{num\_segments}, this indicates 
that the reconstruction will be carried out using segment numbers 
-\textit{num\_segments} through +\textit{num\_segments} of the measured 
projection data. Defaults to use all segments in the input data.


\item[prior type]
The type of prior information. See Section \ref{sec:priors} for the list 
of possible values.

\item[zoom]
The zoom factor. Defaults to 1 which gives x,y voxel size equal 
to the bin size of the projection data\footnote{{\small This is different 
from the CTI convention which involves some factors like the 
number of elements and 128.}}. Larger than 1 means smaller voxel 
size. This parameter is ignored when an \textit{initial estimate} is specified.
\textit{This parameter will be removed in a future version.}

\item[XY output image size (in pixels)]
Number of pixels to use in x,y direction. Default (-1) covers 
the whole FOV as determined from the projection data and gives 
an odd number of pixels. This parameter is ignored when an \textit{initial 
image} is specified. \\
\textit{This parameter will be removed in a future version.}

\end{description}


{ \subsubsubsubsection{Algorithm settings}
}
\label{sec:IterativeReconSettings}
\begin{description}

\item[output filename prefix]
 The output filename prefix. When image iterates are saved, subiteration 
numbers are appended to this prefix after an underscore.

\item[output file format type]
Set output file format, if omitted a default value will be used. 
Subsequent parameters will depend on the type of output file 
format. See section \ref{sec:outputfileformats}..


\item[initial estimate]
The name of the image file with which the algorithm is initialised. 
The default is an image which is uniformly 1 over all voxels. 
The default can be specified by giving the parameter an input 
of 1. Using an input of 0 will use a uniformly 0 image, but this 
is inappropriate for OSMAPOSL.


\item[enforce initial positivity condition]
If this parameter is not set to 0, the program will set all non-positive 
voxel values in the initial estimate to small positive ones.


\item[number of subsets] [1, num\_views, 1 {]}
The number of subsets. For symmetric balancing among subsets, 
it is advisable to select a number which divides evenly into \textit{num\_views},
or even \textit{num\_views/4}, depending on the symmetries used by the projector.


\item[uniformly randomise subset order]
When set to 1, the order of the subsets for each full iteration 
is randomly permuted (uniformly).


\item[start at subset] [0, num\_subsets-1, 0{]}
Specifies with which subset in the ordered subset sequence to 
start, where the subsets are enumerated from \textit{0} to \textit{num\_subsets-1}. 

\item[number of subiterations] [1,\ensuremath{\infty}, 1{]}
The number of subiterations to run.


\item[start at subiteration number] [1,\ensuremath{\infty}, 1 {]}
Initializes the subiteration counter to the input value. Useful 
for resuming reconstructions (for example, the continuity of 
output image file numbering may be preserved).


\item[save estimates at subiteration intervals] [1, num\_subiterations,0{]} 
Specifies at what intervals (in subiterations) the iterates of 
the algorithm are saved. The final iteration is always saved.



\item[inter-iteration filter subiteration interval] [0, num\_subiterations, 0{]}
Specifies at what intervals (in subiterations) inter-iteration 
filtering is carried out. A value of 0 disables inter-iteration 
filtering.


\item[inter-iteration filter type]
The type name of the inter-iteration filter. See Section \ref{sec:filters}
for the list of possible values.

\item[post-filter type]
The type name of the post filter. See Section \ref{sec:filters} for the list 
of possible values.


\item[inter-update filter subiteration interval] [0, num\_subiterations, 0{]}
Specifies at what intervals (in subiterations) inter-update filtering 
is carried out (see [Jac00]). A value of 0 disables inter-update filtering.\\
\textit{(OSMAPOSL specific)}

\item[inter-update filter type]
The type name of the inter-update filter. See Section \ref{sec:filters} for 
the list of possible values.\\
\textit{(OSMAPOSL specific)}


\item[map model : additive{\textbar}multiplicative]
The default choice \textit{additive} applies the prior on the image 
as is, the other choice \textit{multiplicative} applies it essentially 
on the image times the sensitivity image. See [Mus01] for more 
details.\\
\textit{(OSMAPOSL specific)}

\item[maximum relative change] [0, 3.40282e+38, 3.40282e+38{]}
The multiplicative update image will be thresholded from above 
with this value (at every subiteration except the first) \textit{i.e.}, 
before multiplying it with the old image to get the new one. 
The default value does not impose any thresholding (as in strict 
OSMAPOSL). However, we find that when subsets are used, a value 
of about 10 is beneficial.\\
\textit{(OSMAPOSL specific)}


\item[minimum relative change] [0, 3.40282e+38, 0{]}
The multiplicative update image will be thresholded from below 
with this value (at every subiteration except the first).\\
\textit{(OSMAPOSL specific)}


\item[write update image] [0, 1, 0{]}
When this is set to 1, OSMAPOSL will write the multiplicative 
update images every sub-iteration.\\
\textit{(OSMAPOSL specific)}

\end{description}


{
\subsubsubsection{
Notes and technical issues regarding the selection of parameters}
}
\label{sec:OSMAPOSLtechnotes}

\textbf{\textit{Application of filters:}} \\
In addition to the remarks in section \ref{sec:filters} on filtering, 
one should 
note the following


(i) Inter-iteration filtering (or inter-update filtering) generally 
results in resolution that is object dependent and space varying. 
This is probably not what you want. See [Mus04,Mus02] for more 
details.



(ii) If a strictly positive Metz power parameter is chosen, a 
non-trivial Metz filter results whose frequency response possesses 
an amplifying middle frequency band. In this case, it is potentially 
hazardous to choose too small a corresponding FWHM parameter. 
For then, the amplifying mid-band may coincide with the high 
frequencies of the noise components of the image and hence strengthen 
these components. Empirically, we have found that good lesion 
detectability is obtained by IMF-OSEM with Metz powers of 1, 
together with FWHMs of 40\%-75\% the relevant dimensions of the 
lesion (see also [Jac00]).



(iii) All filtering operations in all iterative algorithms currently 
implemented in the library apply post-thresholding 
to ensure positivity at each iteration. Future 
releases may allow users to vary the thresholding rule or de-activate 
it.


\textbf{\textit{Zeroing end planes of segment 0}}: 


The ``zero end planes of segment 0'' option is made available 
to help users of the \textit{STIR} reconstruction software overcome 
a modelling difficulty created by an awkward image discretisation 
convention. So that images reconstructed using the \textit{STIR} software 
can be compared to those of other reconstruction software, the \textit{STIR} 
library defaults to an image discretisation scheme that has become 
common among most commercial analytic reconstruction software 
packages. In this scheme, images are discretised on a voxel grid 
in which there are 2*num\_rings-1 transaxial slices of voxels, 
num\_rings denoting the number of rings in the scanner. The axial 
extent of each slice is half that of the detector rings and the 
extreme-most voxel slices are centered about the mid-plane of 
the extreme-most detector rings. A consequence of this set-up 
is that the voxel grid does not cover the entire length of the 
scanner, but rather leaves a gap at each end of the scanner whose 
axial extent is half the axial length of a detector ring.



We have discovered that this creates a problem for iterative 
MLE algorithms when reconstructing activity distributions that 
extend into these ``end gaps''. For the activity present in the 
gaps is surely detected by the scanner. However, the restricted 
extent of the voxel grid imposes a model of the projection data 
statistics in which the activity in the gaps is zero. As a result, 
the algorithm assigns an excess of activity to the end planes 
of the image grid, seemingly because they are the next most likely 
origins of the annihilations in the end gaps. In turn, the reconstructed 
images exhibit excessively bright end planes.



From a theoretical point of view, there is nothing ``wrong'' with 
such reconstruction results. The image observed approximates 
a maximum likelihood estimate consistent with the parametric 
model assumed. From a practical point of view, however, it is 
obvious that the parametric model assumed is an inferior one, 
since it does not reflect the physical reality that activity 
is present in the end gaps. In turn, poor quantification of the 
end plane activity results. 



We have found that using the ``zero end planes of segment 0'' 
option can often remedy this problem. This causes the reconstruction 
to pretend that the detector pairs in the extreme-most detector 
rings detected no counts. For the scanners currently supported 
by the library, we have verified that only the measurements of 
these detector pairs can be influenced by the end gap activity. 
By setting this option, one models 
the acquisition as one where the end ring detector pairs had 
zero efficiency. This provides a more realistic model for the 
truncated projection data. Note also that the acquisition software 
of some scanners (e.g. the Positron HZL/R) automatically discards 
the end plane data. For the same reasons, it is advisable to 
use the ``zero end planes of segment 0'' in these cases as well.



The method of zeroing end planes, unfortunately, does not apply 
well to scanners whose 2D projection data includes merged cross-planes 
(e.g. the GE scanners or all recent Siemens scanners). 
This is because the currently supported 
projectors yield a relatively poor model of the detection probabilities 
associated with the end planes of such scanners. Excluding the 
end detector pairs from the computation degrades the model still 
further. Consequently, some of the voxel values in the end planes 
may get very large (i.e. noisy) with increasing iterations. For such scanners, we 
advise that users forego the ``zero end planes of segment 0'' 
option. Doing so still yields very reasonable results in the 
interior part of the reconstructed image. Moreover, future development 
of the \textit{STIR} library's suite of projector functions will remedy 
most of the problems discussed in this section.


Finally, for phantom studies in which the activity is known to 
lie within the space covered by the voxel grid, the end plane 
phenomenon does not arise.

{\subsubsubsection{Extra parameters the MPI version}}\\
When running the MPI version of \texttt{OSMAPOSL} on distributed memory
systems, the following extra parameters can be set in the .par files.

\begin{description}
\item[enable distributed caching] (default: 0)
Enables/disables the caching algorithm to save some communication overhead (according
to Tobias Beisel's tests, this makes only makes a difference with very large files). 

\item[enable distributed tests] (default : 0)
Tests to check whether the distributed functions work. This is of no use if you're 
not developing new code for the parallel version. It could be thrown out of the code at some point.

\item[enable message timings] (default : 0)
Prints timings of send and receive operations, to see where the time gets lost. 
Could be used for later optimization. 

\item[message timings threshold] (default : 1.0)
Defines a threshold of the above timings, so that not all messages are timed, but those which 
consume a serious amount of time. 

\item[enable rpc timings] (default : 0)
Measures the total time and the average slave time spend on RPC\_process\_related\_viewgrams\_gradient() function. 
Gives some indication of how much time you save by calculating in parallel. 
\end{description}

See [Bei08] for some info on performance.

{ \subsubsubsection{OSSPS}
}
\label{sec:OSSPS}

  { \subsubsubsubsection{Algorithm description}
  }

  OSSPS is a relaxed preconditioned sub-gradient descent algorithm:

  \[ \lambda^\mathrm{new} = \lambda + \zeta D \nabla \Psi \]

  with $\lambda$ the parameters to be estimated,
  $\Psi$ the objective function,
  $D$ a dagional matrix (the preconditioner)
  and $\zeta$ an iteration-dependent number called the relaxation parameter (see below).

  $D$ depends on $\Psi$. The data-dependent term in the preconditioner 
  suggested in Ahn and Fessler turns out to be equal to the product of the Hessian of 
  $\Psi$ summed over all columns (while using the data plug-in approach for 
  approximating the Hessian close to the solution). Therefore, this OSSPS 
  implementation uses this prescription in general and can be applied to any objective
   function implemented in \textit{STIR}.  

  Note that the exact paraboloidal surrogate algorithm by Erdogan and Fessler is currently
  not implemented.

  The relaxation value for the additive update follows the suggestion from Ahn and Fessler:

 \begin{equation}
  \zeta= { \alpha \over 1+ \gamma n } \label{eq:OSSPSrelaxation}
 \end{equation} 

  Note that because of the nature of the preconditioner, the OSSPS algorithm does not
  perform very well in our experience when there is no ``additive sinogram''.

{ \subsubsubsubsection{Running OSSPS}
}

 Most of what has been described above for OSMAPOSL applies verbatim for OSSPS. 
 See \textbf{STIR/samples/} for example parameter file(s). The
 following describes the parameters that are specific to OSSPS:

 \begin{description}
 \item[precomputed denominator]
   if you do not specify this keyword, the data-dependent part of the
   precomputed denominator will be computed automatically (and saved using
   \texttt{output filename prefix}). You can use this parameter to
   re-use the file from a previous run. \\
   Note: setting the value to 1 will use an images full of ones (which is not a good idea!)

 \item[relaxation parameter]
   $\alpha$ in the formula \ref{eq:OSSPSrelaxation} above
 \item[relaxation gamma]
   $\gamma$ in the formula \ref{eq:OSSPSrelaxation} above

 \item[upper bound]
  you can give an upper bound on the image values (the lower bound is always zero). The upper
  bound defaults to a very large value.
\end{description}

{ \subsubsubsection{Using list mode data as input}
}
\label{sec:ListmodeIterativeAlgorithms}
Since \textit{STIR} 2.0, you can specify an objective function that takes list mode data as input
and uses a projection matrix. This can be used to do a static image reconstruction
without any resolution loss etc. 

\textbf{WARNING: The current implementation has not been thoroughly tested.} Please use
the mailing lists for questions and results.

The following gives the main parameters to use for OSMAPOSL.
\begin{verbatim}
OSMAPOSLParameters :=
  objective function type:=\
     PoissonLogLikelihoodWithLinearModelForMeanAndListModeDataWithProjMatrixByBin
  PoissonLogLikelihoodWithLinearModelForMeanAndListModeDataWithProjMatrixByBin Parameters:=
    list mode filename:= listmodefile.lm
    time frame definition filename:= timeframes.fdef
    ; time frame to use data for the reconstruction
    time frame number:=1
    ; maximum ring difference (defaults to all ring differences)
    ;max ring difference num to process:=
    ; matrix to use for forward and back projection (defaults to ray tracing)
    ;Matrix type:= using ray tracing
    ; background term
    additive sinogram:=
    sensitivity filename:=filename

    ; other usual objective function parameters 

  End PoissonLogLikelihoodWithLinearModelForMeanAndListModeDataWithProjMatrixByBin Parameters:=

  ; normal OSMAPOSL algorithm parameters 

End:=
\end{verbatim}

Currently, the sensitivity filename needs to be specified as the code to compute it using
a bin normalisation type has not yet been copied to the 
PoissonLogLikelihoodWithLinearModelForMeanAndListModeDataWithProjMatrixByBin
class. You could run OSMAPOSL first with projection data without axial compression (and view
``mashing'') to create the sensitivity filename. Similarly, if you want to use OSSPS,
you will need to run the projection data version first to get the ``precomputed denominator''.

The additive sinogram is as discussed in \ref{sec:PoissonProjectionDataObjectiveFunction}, 
but needs to be without axial compression (and view ``mashing'').

{ \subsubsubsection{Parametric image estimation algorithms}
}
\label{sec:ParametricImageIterativeAlgorithms}
Both OSMAPOSL and OSSPS work have implementations to estimate parametric images from
dynamic data, called \textbf{POSMAPOSL} and \textbf{POSSPS} respectively. 
Their parameters are identical to the static versions of the algorithms, except that
they need a different type of objective function. Currently, the only defined version is
where the input is dynamic projection data and the Patlak linear model is used to go from
dynamic images to parametric images. See section {sec:KineticModels} for more information.

Currently, the input and output has to be in ECAT7 format. 

Below is an excerpt of a possible parameter for file POSMAPOSL. A complete parameter file
can be found in \textbf{STIR/samples/}. The only change is the fact that dynamic data
form the input, and that a (linear) kinetic model has to be specified.

\begin{verbatim}
OSMAPOSLParameters :=
  objective function type:=\
     PoissonLogLikelihoodWithLinearKineticModelAndDynamicProjectionData
  PoissonLogLikelihoodWithLinearKineticModelAndDynamicProjectionData Parameters:=

  ; input dynamic projection data with emission
  input file := SOME_DYNAMIC_PROJDATA.S
  ; specify additive (dynamic) projection data to handle randoms or so
  additive sinograms := SCATTER_AND_RANDOMS_ADDED_ECAT7.S

  ; kinetic model specification
  Kinetic Model type := Patlak Plot
    Patlak Plot Parameters :=
    ;; see the Patlak model description elsewhere
    end Patlak Plot Parameters :=

  ; other parameters such as projectors and sensitivity 
  ; identical to the static case

  end PoissonLogLikelihoodWithLinearKineticModelAndDynamicProjectionData:=

  ; normal OSMAPOSL algorithm parameters 

End:=
\end{verbatim}

\subsubsection{
Filtered back projection (FBP2D)}

This implements SSRB+FBP. Currently, data have to be completely 
precorrected before-hand (arc-correction will be performed automatically
if necessary).\\
The implementation is careful about the implementation of the 
ramp-filter to avid problems with the DC component (see RampFilter.cxx 
for more details).



As described in Section \ref{sec:conventions}, the recommended manner of running \textbf{FBP2D}  
is to pass the executable a parameter file argument which specifies 
the relevant parameters of the reconstruction. Parameters omitted 
from the file are assigned a default value. If a parameter file 
is not used, the program prompts the user for the required information.


A sample.par file can be found in the \textbf{samples/} directory. 
Below are the parameters you probably need.


\begin{verbatim}
fbp2dparameters :=

input file := input.hs
output filename prefix := output

; output image parameters
; zoom defaults to 1
zoom := 1
; image size defaults to whole FOV
xy output image size (in pixels) := 180

; can be used to call SSRB first
; default means: 
; if no axial compression, use 3 
; otherwise, use 1
;num segments to combine with ssrb := -1

; filter parameters, default to pure ramp
alpha parameter for ramp filter := 1
cut-off for ramp filter (in cycles) := 0.5

; keywords that specify the filtering that occurs at the end
; of the reconstruction
post-filter type := None

end := 
\end{verbatim}

\textbf{Warning:} the current version of the interpolating backprojector, 
the default backprojector used by FBP2D, has a central artefact 
on some systems (including Sparc and Opterons). 


\subsubsection{
3D Reprojection Algorithm (FBP3DRP)}

This implements Kinahan and Rogers FBP algorithm with reprojection 
of the missing data [Kin89]. The implementation is fairly generic 
and should be able to handle non-standard data (e.g. with less 
or more sinograms in a segment than you would normally have). 
Also, the Colsher filter is numerically computed at a finer grid 
in an initial stage to avoid DC problems.


A sample.par file can be found in the \textbf{samples/} 
directory. See separate documentation on FBP3DRP, the doxygen documentation 
(or the source) for more info.\\
\textbf{Warning:} the current version of the interpolating backprojector, 
the default backprojector used by FBP3DRP, has a central artefact 
on some systems (including Sparc and Opterons).

\subsection{
Scatter Correction}
Coincidence events where one or more of the photons are scattered form a major contribution to the data
in 3D PET. This scatter background needs to be estimated and taken into account during reconstruction. In
\textit{STIR 2.1}, a version of the Single Scatter Simulation ( \textit{SSS}) algorithm [Wat96, Wat00,
Wat97, Wat04, Wer02, Poe03] has been implemented and evaluated [Tso04].

\subsubsection{Data Initialisation}
\begin{itemize}
\item Apply randoms correction, normalisation and attenuation correction on emission data 
\item Reconstruct attenuation image
\item Use \textbf{create\_tail\_mask\_from\_ACFs} to estimate the region that will be used for scaling
\item Reconstruct emission image without scatter correction
\item Construct a "sub-sampled sinogram" template (i.e. same geometrical characteristics but
less rings and detectors without arc-correction) as suggested in literature (e.g. in [Tso04]).
\item Reconstruct a coarse attenuation image or else construct a "subsampled" attenuation image using zoom\_image and stir\_math. 
\end{itemize}

\cmdline{%
create\_tail\_mask\_from\_ACFs		--ACF-filename <projdata> \textbackslash \\
					--ACF-threshold <number (1.01)> \textbackslash \\
					--output-filename <projdata> \textbackslash \\
					--safety-margin <0>%
}
\begin{description}
\item[ACF-filename] Attenuation Correction Factors Projection Data Filename
\item[ACF-threshold] Value used to threshold ACF and creates the mask, defaults to 1.01
\item[safety-margin] This dilates the mask for a number of safety bins in tangential direction
\item[output-filename] The file that new projection data will be stored
\end{description}

\subsubsection{SSS}
\begin{itemize} 
\item Use \textbf{estimate\_scatter} to find subsampled scatter estimate from current emission image
\item Use \textbf{upsample\_and\_fit\_single\_scatter}
\end{itemize} 


\cmdline{estimate\_scatter scatter.par}
Example \texttt{scatter.par}
\begin{verbatim}
Scatter Estimation Parameters:=
; threshold below which not to bother putting a scatter point in a voxel
   attenuation_threshold :=.01
; Placing the scatter points in a random position of a voxel (0) else randomly so
   random := 1
; Caching is extremely useful as it speeds up the estimation drastically.
   use_cache := 1
; The resolution of the scanner as given in literature
   energy_resolution :=.22
; Acquisition settings
   lower_energy_threshold :=350
   upper_energy_threshold :=650
   activity_image_filename := ACTIVITY_IMAGE
   density_image_filename := DENSITY_IMAGE
   density_image_for_scatter_points_filename := LOW_RESOLUTION_DENSITY_IMAGE
   template_proj_data_filename := SUBSAMPLED_PROJECTION_DATA_TEMPLATE
   output_filename_prefix := OUTPUT_PROJECTION_DATA
End Scatter Estimation Parameters:=
\end{verbatim}

\cmdline{%
upsample\_and\_fit\_single\_scatter \textbackslash \\
					--min-scale-factor <number> \textbackslash \\
					--max-scale-factor <number> \textbackslash \\
					--remove-interleaving <1|0> \textbackslash \\
					--half-filter-width <number> \textbackslash \\
					--output-filename <filename> \textbackslash \\
					--data-to-fit <filename> \textbackslash \\
					--data-to-scale <filename> \textbackslash \\
					--weights <filename>%
}
\begin{description}
\item[min-scale-factor,max-scale-factor]: Both min and max scale factors are set such that the scaling does not explode for any reason.
\item[remove-interleaving]: Remove the interleaving in the coarse scatter estimate before upscaling (it is recommended to set this to 1)
\item[half-filter-width]: width of filter to apply on estimated scale factors (in axial direction). The filter is a simple boxcar.
\item[output-filename]: Scatter projection data at the emission data settings
\item[data-to-fit]: Emission sinogram 
\item[data-to-scale]: Course scatter sinogram 
\item[weights]: This is the mask sinogram, or any other weighting factors on projection space that is used to define the region of scaling. 
\end{description}
The procedure of scaling involves B-Splines interpolation for the tangential (Cubic), axial (Cubic) and
view (Linear) dimensions, while for oblique segments, currently the interpolation is achieved by applying the
pseudo-inverse transform of Single Slice Rebinning as described in [Tso05].

\subsubsection{Iterative loop (if necessary)}
The scatter estimation is based on images that do not include scatter, therefore an iterative procedure
is needed such that scatter is eliminated progressively. In the first iteration, scatter (B.2) is
overestimated and thus images are overcorrected (B.3) for it. Then, in the second iteration (B.2) scatter
is underestimated. The mean value of the two scatter estimates can be used for scatter correction. Two
iterations are usually enough to estimate a good approximation of scatter.

\begin{itemize}
\item Reconstruct emission image with scatter estimate\\
Two options exist:
\begin{itemize}
\item	FBP: Correct the scatter estimate for attenuation and subtract from the precorrected emission data. 
Run FBP with sufficient filtering.
   \[\mathrm{Image} = \mathrm{FBP}\left(\mathrm{ACF}*(N*(P-R)-S)\right) \]
\item Iterative (e.g. OSEM or OSSPS): Use the following for the additive sinogram (\textit{A}).
    \[A=(R/N+S)/(\mathrm{ACF})  \]
\end{itemize}
where, \textit{N}: Normalization Factors, 
	\textit{ACF}: Attenuation Correction Factors,
	\textit{R}: Random Projection Data,
	\textit{S}: Scatter Projection Data as estimated using  \textit{STIR},
	\textit{P}: Measured Projection Data (prompts)
\end{itemize}

\subsubsection{Limitations}
The three separate executables need to be merged in one together with the iterative loop. The
\textit{SSS} can estimate 3D scatter that is measured by indirect rings but no interpolation of indirect
rings is currently in place. Therefore, upsampling of oblique sinograms is currently handled in a simple
way i.e. (pseudo)-inverse SSRB but ideally interpolation of the 3D sinograms will be useful. In the
current version, we have only \textit{SSS} in \textit{STIR} with scaling per sinogram of the scatter
sinograms for taking consideration of the out of field of view activity and multiple scattering
events. This generally seems to produce good results in most instances [Thi07].


\subsection{
Parametric Image Construction using kinetic modelling}
\label{sec:KineticModels}
A parametric image class (\textbf{ParametricDiscretisedDensity}) has been implemented that at the moment
can hold two parameters\footnote{Most relevant \textit{STIR} classes are templated in the number of parameters
and would therefore work with numbers. There are some preprocessor defines however to shorten the code in
a few places. Currently this uses \texttt{\#DEFINE NUM\_PARAMS 2}.}. 
In this release \textit{Patlak plot} [Pat83, Pat85], i.e. a linear kinetic
model for tracers with irreversible kinetic behavior, has been implemented. The \textbf{PatlakPlot} class
is derived from the \textbf{KineticModel} class.

The user will need to provide information about:
\begin{itemize} 
\item	File of input function (e.g. plasma activity or reference tissue activity values); 
\item	Calibration Factor (i.e. scale factor that sets the projection data to the same units with the input function) 
\item	Starting frame to apply  \textit{Patlak Plot} 
\item	Delay between input function and PET measurement 
\item	Time-frame definition file. 
\item	The boolean \textit{In total counts} is 0 or 1 depending if the counts are total or mean over each frame 
\item	In correct scale should be set to 1 if  \textit{STIR} has already scaled correctly the images according to \textit{x\_voxel\_size}
\end{itemize}
Example Sample File: \texttt{PatlakPlot.par} 

\begin{verbatim} 
Patlak Plot Parameters:=

time frame definition filename := 
starting frame := UNSIGNEDINT
calibration factor := FLOAT
scale factor := FLOAT
blood data filename := plasma.if
; In seconds
Time Shift := FLOAT
In total counts := 1
In correct scale := 1

end Patlak Plot Parameters:=
\end{verbatim}

Current assumptions are that an \textit{F-18} tracer is used; the input function is not decay corrected and it
is given in a text file with three columns (i.e. \textbf{PlasmaSample}): time in \textit{seconds}, plasma-activity in
\textit{kBq/cm-3}, and blood-activity in \textit{kBq/cm-3}) 
%HARRY TODO "and multiple" ?
and multiple the total number of \textbf{PlasmaSample}
is usually given in the very first row. Currently, the parametric image
has to be in \textbf{ECAT7} format with \textbf{num\_frames} being equal to \textbf{num\_params}.

How to Run \textit{Patlak Plot}:

\cmdline{%
apply\_patlak\_to\_images \textbackslash \\
  output-parametric-image input-dynamic-image patlak-plot.par}

How to Estimate Dynamic Images using \textit{Patlak Plot} Parametric Images:
\cmdline{%
get\_dynamic\_images\_from\_parametric\_images \textbackslash \\
   output-dynamic-image input-parametric-image patlak-plot.par}

Of course the reference tissue model can also be applied: 

\begin{verbatim}
Patlak Plot Parameters:=

time frame definition filename := 
starting frame := UNSIGNEDINT
calibration factor := 1
scale factor := 1
blood data filename := tac.roi
; In seconds
Time Shift := 0
In total counts := 1
In correct scale := 1

end Patlak Plot Parameters:=
\end{verbatim}

\textbf{Important Note}: The current version of indirect reconstruction does not consider any special weighting for the time frames. This is a good approximation only if the counts at each frame is very similar which is mostly true for frames with the same duration. If the user needs to use  \textit{weighted least-squares}, the code will need minor modification.

Moreover, two direct reconstruction version of linear kinetic models have been also implemented, evaluated [Tso08, Tso07] and optimised [Ang11]. These are the parametric versions of \textbf{OSEM} and \textbf{OSSPS} therefore called \textbf{POSEM} and \textbf{POSSPS}. 
P
\cmdline{POSMAPOSL POSEM.par}
\cmdline{POSSPS POSSPS.par}


The parameter files include the information of the kinetic model as described for the indirect case and the declaration of the new objective function type\\
\texttt{PoissonLogLikelihoodWithLinearKineticModelAndDynamicProjectionData}.

%TODO More information could be included for \textbf{MAP} estimation e.g. \textbf{quadratic priors}. 

\textbf{Important Note}:  \textit{Patlak plot} Parameters are highly correlated thus \textbf{POSEM} and \textbf{POSSPS} will struggle to compute the parameters [Tso07]. However, for (non-negative) kinetic models that correlation is minimum it is expected that they will work fine.


\subsection{
Utilities}

Programs are given in the \textbf{STIR/utilities} directory, that 
allow the user to display, manipulate and convert \textbf{interfile} 
data, either image or projection data.

Utilities specific to files in ECAT 6 or 7 format or in the \textbf{ecat}
sub-directory.

\subsubsection{
Displaying and performing operations on data}

The programs are \textbf{manip\_image}, \textbf{list\_image\_info}, 
\textbf{list\_image\_values} and \textbf{find\_maxima\_in\_image}
for image data and \textbf{manip\_projdata}, 
 \textbf{display\_projdata}, \textbf{extract\_segments}, 
\textbf{list\_projdata\_info} for 
projection data and \textbf{stir\_math} for both. Run them with the 
name of the image or projection data file as an argument. For 
example, for \textbf{manip\_image}:
\cmdline{manip\_image file\_name.hv}


See also section \ref{sec:SSRB} for the use of \textbf{SSRB} to manipulate projection 
data.


The program \textbf{get\_time\_frame\_info} is \textit{STIR}'s first (small!) 
step into supporting dynamic data.


\textbf{Note}: The displaying functionality is \textit{really} basic, and 
only intended for a quick check how your data looks like. For 
any serious work, use a decent viewer. For example, \textit{AMIDE} 
is a free viewer that can read \textit{STIR} data
\footnote{\R2Lurl{http://amide.sourceforge.net }{http://amide.sourceforge.net}}. Alternatively, 
convert your \textit{STIR} data to another format using (X)medcon
\footnote{\R2Lurl{http://xmedcon.sourceforge.net }{http://xmedcon.sourceforge.net}}.

{ \subsubsubsection{manip\_image}
}

This program works on two modes. Additionally to the display 
possibility, the main mode allows to retrieve information about 
the image:

\begin{itemize}
\item the minimum and maximum values; either for a plane or for an 
entire 3D image.
\item the number of counts.
\item the voxel value profile; either for an image slice or for any 
row 1D row through the image.
\end{itemize}
Within this mode, a truncation (here called trim) of the data 
can be performed; it results in the circular resetting to zero 
of pixels at the image edges.


The other mode (math mode) lets one perform arithmetic operations 
between two images, or between an image and a scalar. The result 
of each calculation is kept in a math mode buffer which becomes 
the input to the next calculation. Hence, a sequence of mathematical 
operations can be carried out on the input image within math 
mode. The math buffer contents can also be displayed inside the 
mode.

{ \subsubsubsection{manip\_projdata}
}

This utility is the counterpart of manip\_image for projection 
data. A menu is displayed with the following options:

\begin{itemize}
\item viewgram-wise / sinogram-wise display.
\item computation of minimum and maximum values and total counts.
\item arithmetic operations between two projection data arrays or 
between a projection data array and a scalar.
\item binarisation of the sinogram, positive values are set to 1, 
negative values to 0.
\item truncation of the negative values.
\item application of a tangential truncating window (e.g. all data 
for bins greater than a specified distance from the scanner axis 
may be set to 0).
\item application of an axial truncating window to segment 0 (e.g. 
end plane data may be set to zero).
\end{itemize}
Each time an operation results in a new projection data array, 
the user is prompted for the name of an output file, to which 
the result is written. This output file then automatically becomes 
the input file to the next selected operation. Similar to manip\_image, 
this allows new projection data files to be generated from a 
sequence of operations.


\textbf{Warning}: the user must ensure that all input and output projection 
data arrays in a given operation are read from / written to separate 
files.

{ \subsubsubsection{display\_projdata}
}

This utility should be preferred to \textbf{manip\_projdata} when the 
goal is to display the data by view or by segment for a defined 
segment number (ring difference).

{ \subsubsubsection{list\_projdata\_info}
}

A utility that lists size info of the projection data on stdout. 
Use this if you think there is something wrong with how \textit{STIR} 
reads your projection data.

{ \subsubsubsection{create\_projdata\_template}
}

This utility is mainly useful to create a template that can then 
be used for other \textit{STIR} utilities (such as fwdtest, lm\_to\_projdata 
etc.).
\cmdline{create\_projdata\_template output\_filename}

This will ask questions to the user about the scanner, the data 
size, etc. It will then output new projection data (in Interfile 
format). However, the binary file will not contain any data.

{ \subsubsubsection{extract\_segments}
}

This utility extracts projection data by segment into a sequence 
of 3d image files. It is mainly useful to import segments into 
external image display/manipulation programs which do not understand 
3D-PET projection data, but can read Interfile images.


The user will be asked if the images should correspond to SegmentByView 
or SegmentBySinogram data. In the first, data are stored as a 
stack of viewgrams (one for each view), in the second as a stack 
of sinograms (one for each axial position). (See the \textit{STIR} glossary).

{ \subsubsubsection{stir\_math}
}
\label{sec:stir_math}

This is a command line utility for adding or multiplying data 
and other numerical operations, with a somewhat awkward syntax. 
Just execute \texttt{stir\_math} for a usage message.

Since \textit{STIR} 2.1, this utility also supports parametric or dynamic data using
the command line switches \texttt{--parametric} and \texttt{--dynamic}.

\textit{Examples}
\begin{itemize}
\item
Adding 3 images
\cmdline{stir\_math output in1 in2 in3}

Sets \textit{output=in1+in2+in3}\\
\item
Subtracting 2 images
\cmdline{stir\_math --times-scalar -1 output in1 in2}


Sets \textit{output=in1-in2}\\
\item
Sum the square of each images
\cmdline{stir\_math --power 2 --including-first output in1 in2}


Sets \textit{output=in1}$^{\mathit{2}}$ \textit{+ in2}$^{\mathit{2}}$\\
\item
Dividing 2 projection data files
\cmdline{stir\_math -s --mult --power -1 output in1 in2}


Sets \textit{output=in1/in2}\\
\item
Dividing 2 projection data files avoiding division by 0 by thresholding 
the 2$^{nd}$ data-set.
\cmdline{stir\_math -s --mult --min-threshold .1 --power -1 {\textbackslash}\\
output in1 in2}


Sets \textit{output=in1/max(in2,.1)}\\
\item
Dividing 2 files, with first file set to the quotient
\cmdline{stir\_math --accumulate --mult --power -1 in1 in2}


Sets \textit{in1=in1/in2}\\
\item
Linear combination of 3 files
\cmdline{stir\_math --times-scalar 5 --divide-scalar 2.5 output {\textbackslash}\\
in1 in2 in3}


Sets \textit{output=in1+2*in2+2*in3}
\end{itemize}

\textbf{Warning} There is no check that the data sizes and other info 
are compatible and the output will have the largest data size 
in the input, and the characteristics (like voxel-size or so) 
are taken from the first input data. Hence, lots of funny effects 
can happen if data are not compatible.


\textbf{Warning} When \texttt{'--accumulate'} is not used, the output file 
HAS to be different from all the input files.

\textbf{Warning} The result of using non-integral powers on negative 
numbers is probably system-dependent.

{ \subsubsubsection{generate\_image}
}

This program can be used to generate 
images containing geometric shapes such as cylinders, spheres 
etc. See the online documentation generated by doxygen for more 
info.

{ \subsubsubsection{zoom\_image}
}

This can be used to reinterpolate 
images to different voxel sizes and/or dimensions. See the online 
documentation generated by doxygen for more info.

{ \subsubsubsection{get\_time\_frame\_info}
}
\label{sec:get_time_frame_info}
This simple program allows display of time frame info for a given 
file. The specified file can be an ECAT6 or an ECAT7 file, or 
a simple text file specifying the number of time frames and their 
durations. See the class documentation for \textit{TimeFrameDefinitions} 
for the format of this text file.


\noindent
Basic Usage:
\cmdline{get\_time\_frame\_info filename frame\_number}


Using no arguments will give a more extensive usage message showing 
some options to select which data to print.

{ \subsubsubsection{list\_ROI\_values}
}

This is a test-release of a program that can be used to find 
ROI values for an image. See the online documentation generated 
by doxygen for more info.


\subsubsection{
Converting data}
\label{sec:convertingdata}
This program is used to convert CTI ECAT 6 data (either image 
or projection data) into interfile data. It normally should be 
run as follows
\cmdline{convecat6\_if output\_file\_name\_without\_extension {\textbackslash}\\
cti\_data\_file\_name [scanner\_name]}


The optional \textit{scanner\_name} can be used to force to a particular 
scanner (ignoring the system\_type in the main header). \textit{scanner\_name} 
has to be recognised by the Scanner class (see STIR/buildblock/Scanner.cxx). 
Examples are: ECAT 953, ART, ECAT HR+, Advance etc. If the \textit{scanner\_name} 
contains a space, the scanner name has to be surrounded by double 
quotes  when used as a command line argument.


If there are no command line parameters, the user is asked for 
these parameters instead. 


The program asks if all frames should be written or not. If so, 
all sinograms/images are converted for a fixed 'data' number. 
For each data set, a suffix is added to the output\_filename 
of the form \texttt{\_f\#g\#b\#d\#} where the \# are replaced 
by the corresponding number of the frame, gate, bed, data.



\textbf{Warning} CTI ECAT 6 files seem to have a peculiarity that 
frames and gates are numbered from 1, while bed positions are 
numbered from 0. Similarly, the number of bed positions in the 
main header seems to be 1 less than the actual number present. 
This is at least the case for single bed studies. If this is 
not true for multi-bed studies, the code would have to be adapted.


\textbf{Warning} Most of the data in the ECAT 6 headers is ignored 
(except dimensions)


\textbf{Warning} Data are multiplied with the subheader.scale\_factor, 
In addition, for emission sinograms, the data are multiplied 
with subheader.loss\_correction\_fctr (unless the loss correction 
factor is \texttt{<} 0, in which case it is assumed to be 1).


\textbf{Warning} Currently, the decay correction factor is ignored 

\textbf{Warning} Note that sinogram data have to be 'corner-swapped', 
see section \ref{sec:ecat_swap_corners}.

{ \subsubsubsection{conv\_to\_ecat6}
}

This program is used to convert image or projection data into 
CTI ECAT 6 data (input can be any format currently supported 
by the library). It normally should be run as follows:


for images:
\cmdline{conv\_to\_ecat6 [-k] [-i] outputfilename.img {\textbackslash}\\
input\_filename1 [input\_filename2 \dots ] scanner}


for projection data:
\cmdline{conv\_to\_ecat6 -s[2] [-k] [-i] outputfilename.scn {\textbackslash}\\
input\_filename1 [input\_filename2 \dots ]}


If there are no command line parameters, the user is asked for 
the filenames and options instead. Unless the --\textit{i} option is 
used, the data will be assigned a frame number in the order that 
they occur on the command line.


See \textbf{STIR/buildblock/Scanner.cxx} for supported scanner names, 
but examples are ``ECAT 953'', ``ART'', 
``Advance''. 


{ \subsubsubsubsection{Command line options}
}\begin{description}
\item[-s2] This option forces output to 2D sinograms (ignoring higher 
segments).
\item[-k] the existing ECAT6 file will NOT be overwritten, but added 
to. Any existing data in the ECAT6 file with the same \texttt{<}frame,gate,data,bed\texttt{>} 
specification will be overwritten.
\item[-i] ask for \texttt{<} frame,gate,data,bed\texttt{>} for each dataset
\end{description}

Note that to store projection data in ECAT6, a 3D sinogram cannot 
be axially compressed (CTI \textit{span}=1).\\
\textbf{Warning} This utility does \textit{not} corner-swap 3D projection 
data back to the `raw' convention, see section \ref{sec:ecat_swap_corners}.

{ \subsubsubsection{conv\_to\_ecat7}
}

This program is used to convert image or projection data into 
CTI ECAT7 data (input can be any format currently supported by 
the library). \textbf{conv\_to\_ecat7} uses the Louvain la Neuve ecat 
library, see section \ref{sec:ECAT67support}. This means it will only work on those 
systems supported by that library. It normally should be run 
as follows


for images:
\cmdline{conv\_to\_ecat7 output\_ECAT7\_name input\_filename1 {\textbackslash}\\
{[}input\_filename2 \dots ] scanner}


for emission projection data
\cmdline{conv\_to\_ecat7 --s output\_ECAT7\_name input\_filename1 {\textbackslash}\\
{[}input\_filename2 \dots ]\tab  }


for sinogram-attenuation data
\cmdline{conv\_to\_ecat7 -a output\_ECAT7\_name orig\_filename1 {\textbackslash}\\
{[}orig\_filename2...]}


If there are no command line parameters, the user is asked for 
the filenames and options instead. The data will be assigned 
a frame number in the order that they occur on the command line.


See \textbf{buildblock/Scanner.cxx} for supported scanner names, but 
examples are ``ECAT 953'', ``ART'', ``Advance''. 


{ \subsubsubsection{ifheaders\_for\_ecat7: ECAT7 support for reading}
}

The current release includes some support for making Interfile 
headers that point towards and ECAT 7 file. This is possible 
because ECAT7 normally stores the data with single subheaders 
per frame/gate/bed/data. \textbf{ifheaders\_for\_ecat} uses the Louvain 
la Neuve ecat library, see section \ref{sec:ECAT67support}. 
This means it will only work on those systems 
supported by that library. \\
This program writes Interfile headers that 'point into' an ECAT 
7 file. That is, the binary data are NOT rewritten. So, the result 
of this program is a collection of Interfile headers for every 
data set in the ECAT 7 file. They are called \textit{ecat7\_filename\_extension\_f1g1d0b0.* 
etc}, indicating which frame, gate, bed, data number the dataset 
corresponds to. Run the program as follows
\cmdline{ifheaders\_for\_ecat7 ecat7\_filename.extension}


This only works with some CTI file\_types. In particular, it 
does NOT work with the ECAT6-like file\_types, as then there 
are subheaders 'in' the datasets.\\
\textbf{Warning} This utility does \textit{not} take corner-swapping of 
3D projection data into account, see section \ref{sec:ecat_swap_corners}.\\
Note that you do not have to use this utility if you want to 
read only ``frame 1, gate 1, data 0, bed 0''. In this case, you 
can pass the ECAT7 file directly to any \textit{STIR} program.
\\
\textbf{Warning} The calibration factor field in the main header of ECAT7 images
is currently ignored.

{ \subsubsubsection{ecat\_swap\_corners}
}
\label{sec:ecat_swap_corners}

{ \subsubsubsubsection{Usage}
}
\cmdline{ecat\_swap\_corners out\_name in\_name}

{ \subsubsubsubsection{What does it do?}
}

For some historical reason, CTI scanners store 3D sinograms sometimes 
in a 'corner-swapped' mode. What happens is that some corners 
of the positive and negative segments are interchanged. (As a 
consequence, segment 0 is never affected).\\
Below is a summary of what Kris Thielemans understood about corner-swapping 
from various emails with CTI people. However, he might have totally 
misunderstood this, so beware!



For ECAT6 data, corner-swapped mode occurs \textbf{\textit{always}} for 
data straight from the scanner. However, data which have been 
normalised using the import\_3dscan utility from CTI are already 
corner-swapped correctly. Unfortunately, there is no field in 
the ECAT6 header that allows you to find out which mode it is 
in.



For ECAT7 data, the situation is even more confusing. Data acquired 
directly in projection data have to be corner-swapped when the 
acquisition was in 'volume-mode' (i.e. stored by sinograms), 
but NOT when acquired in 'view-mode' (i.e. stored by view). It 
seems that bkproj\_3D\_sun follows this convention by assuming 
that any ECAT7 projection data stored in 'volume-mode' has to 
be corner swapped, and when it writes projection data in 'view-mode', 
it does the corner swapping for you. So, although there is strictly 
speaking no field in the ECAT7 header concerning corner swapping, 
it seems that the `storage mode' field determines the corner swapping 
as well.\\
When the data is acquired in listmode, this changes somewhat. 
Apparently, there is a parameter in the set-up of listmode scans 
that allows you to put the ACS in 'volume-mode' or 'view-mode'. 
The resulting listmode files encode the sinogram coordinates 
then with corner-swapping or without. After the acquisition, 
the listmode data has then to be binned into projection data. 
It is then up to the binning program to take this corner-swapping 
into account. This is easiest to do by generating 'volume-mode' 
projection data when a 'volume-mode' when the listmode setup 
was in 'volume-mode', and similar for 'view-mode'.


If this sounds confusing to you, KT would agree. Here seems to 
be the best thing to do:


\textit{Do all acquisitions in 'view-mode', set-up your listmode 
scan in 'view-mode', bin the data in 'view-mode'. Forget about 
corner-swapping.}


If you cannot do this, then this utility will corner-swap the 
projection data for you.

{ \subsubsubsubsection{Who implemented this and how was it tested?}
}

The actual corner swapping code was supplied by Christian Michel, 
based on code by Larry Byars. KT has tested it by performing 
a very long cylinder scan in 'volume-mode' on the ECAT 966, and 
looking at the delayeds. The oblique segments had obvious discontinuities 
in the efficiency patterns. After applying this utility, these 
discontinuities appeared.



\textbf{Warning} This utility does not (and cannot) check for you 
if the data has to be corner-swapped or not. So, it can do the 
wrong thing.

{ \subsubsubsection{copy\_ecat7\_header}
}
Allows copying header info between ECAT7 files. Check doxygen, or run without 
parameters for usage info.

{ \subsubsubsection{conv\_AVW}
}

\label{sec:convAVW}
If you have the \textit{AVW}\texttrademark  library
\footnote{See \R2Lurl{http://www.mayo.edu/bir/Software/AVW/AVW1.html}
{www.mayo.edu/bir/Software/AVW/AVW1.html}.
} installed on your system, the make process should have built \texttt{utilities/conv\_AVW}.
See also \ref{sec:AVWsupport}.
This utility allows to use the \textit{AVW} library to read an image, and then write it out
using \textit{STIR} as Interfile.\\
\textbf{Warning:} the \textit{AVW} library seems to do flip some images depending
on the file format. For instance, it reads \textit{ECAT7} files using a z-flip compared
to \textit{STIR}.

It normally should be run as follows:
\cmdline{conv\_AVW [ --flip\_z ] imagefile}

It will require access to a run-time license for \textit{AVW}.


\subsubsection{
Filtering image data}

The \textbf{postfilter.cxx} program allows one to apply any available 
image processor on an input image. \footnote{{\small The current version 
does no longer allow to compute the impulse response (i.e. point 
spread function) of a discretised filter. This is largely because 
the image processors can be non-linear, in which case the PSF 
concept does not apply. For linear filters, it would still be 
possible to obtain the PSF by using as input image an image with 
all 0s except a single pixel in the middle of the image.}} Review 
Section \ref{sec:filters} for info on which filters you can apply.


The program is run as follows:
\cmdline{postfilter [\texttt{<}output filename\texttt{>} [\texttt{<}input file 
name\texttt{>} {\textbackslash}\\
{[}\texttt{<}postfilter.par filename\texttt{>}]]]}


where the square brackets denote optional parameters (their value 
will be asked interactively). Example postfilter.par files can 
be found in the \textbf{STIR/samples} directory.


\subsubsection{
Comparing files}

The two utilities \textbf{compare\_image} and \textbf{compare\_projdata} 
can be used to see if 2 files are identical up to rounding errors. 
(Note that running reconstructions on a different architecture, 
or even when using different compilers will almost certainly 
give rounding error differences.) They should be run with 2 command 
line arguments, specifying the 2 filenames. Optional arguments 
are as follows:
\cmdline{compare\_projdata file1 file1 
[maximum\_segment\_number\_to\_process]}
\cmdline{compare\_image [-r rim\_truncation\_in\_pixels] file1 file1}


where the \texttt{rim\_truncation} argument to \textbf{compare\_image}  says 
how many pixels it should ignore at the radial rim of the image.



\subsubsection{
Precorrecting (or uncorrecting) projection data}

The \textbf{correct\_projdata} utility located in STIR/\textbf{utilities} 
is useful to perform precorrections such as randoms and/or scatter 
subtraction, normalisation and attenuation correction. It can 
also be used to 'uncorrect' the data which might be useful if 
you get completely precorrected data out from the scanner (or 
FORE) and want to reverse some of the corrections.


It is run as
\cmdline{correct\_projdata \texttt{<}correct\_projdata.par filename\texttt{>}}


A sample parameter file is given in \textbf{STIR}/\textbf{samples} and 
is more or less as follows

\begin{verbatim}
correct_projdata Parameters := 
input file := trues.hs

; Current way of specifying time frames, pending modifications to
; STIR to read time info directly from the headers

; The specified file can be an ECAT6 or an ECAT7 file, or a simple
; text file. See also section \ref{sec:get_time_frame_info}. 
time frame definition filename := frames.fdef

; if a frame definition file is specified, you can say that 
; the input data corresponds to a specific time frame
; the number should be between 1 and num\_frames and defaults to 1
; this is currently only used to pass the relevant time 
; to the normalisation 
time frame number := 1

; output file
; for future compatibility, do not use an extension in the name of 
; the output file. It will be added automatically 
output filename := precorrected

; default value for next is -1, meaning 'all segments' 
; maximum absolute segment number to process :=  

; use data (1) or set to one (0) := 

; apply (1) or undo (0) correction := 

; parameters specifying correction factors 
; if no value is given, the corresponding correction will
; not be performed

; random coincidences estimate, subtracted before anything else 
; is done
; randoms projdata filename := random.hs 

; normalisation (or binwise multiplication) 
Bin Normalisation type := from projdata 
Bin Normalisation From ProjData := 
normalisation projdata filename:= norm.hs 
End Bin Normalisation From ProjData:=

; scatter term to be subtracted AFTER norm(+atten correction)
; defaults to 0 
;scatter projdata filename := scatter.hs
END:=
\end{verbatim}


Time frame definition is only necessary when the normalisation 
type uses this time info for dead-time correction (which is not 
supported yet in the current (public) version of \textit{STIR}).


The following gives a brief explanation of the non-obvious parameters. 

\begin{description}

\item[use data (1) or set to one (0)]
Use the data in the input file, or substitute data with all 1's. 
This is useful to get correction factors only. Its value defaults 
to 1.

\item[apply (1) or undo (0) correction]
Precorrect data, or undo precorrection. Its value defaults to 
1.

\item[Bin Normalisation type]
Normalisation (or binwise multiplication, so can contain attenuation 
factors as well). See Section \ref{sec:binnormalisation}.


\item[attenuation image filename] \textbf{obsolete}
Specify the attenuation image, which will be forward projected 
to get attenuation factors. Has to be in units \textit{cm}$^{\mathit{-1}}$.


This parameter will be removed. Use instead a \textit{chained} 
bin normalisation (section \ref{sec:chainedbinnormalisation}) 
with a bin normalisation 
\textit{from attenuation image} (section \ref{sec:binnormalisationfromattenuationimage}).


\item[forward\_projector type] \textbf{obsolete}
Forward projector used to estimate attenuation factors, defaults 
to Ray Tracing. See Section \ref{sec:forwardprojectors}.

This parameter will be removed.
\end{description}

The \textbf{calculate\_attenuation\_coefficients} utility (located in STIR/\textbf{utilities})
can also convert an attenuation image to attenuation (correction) factors. It is less
flexible than \textbf{correct\_projdata} though.

The \textbf{attenuation\_coefficients\_to\_projections} utility 
(located in STIR/\textbf{utilities}) is a (simplistic) example to convert
attenuation (correction) coefficients to projections by taking the logarithm.
For real-world use, it is recommended to pre-filter the data first, to reduce noise
and remove negatives.


\subsubsection{
Generating Poisson noise}

For simulation purposes, it is often useful to be able to generate 
multiple noise realisations given the `true' mean projection data. 
For PET and SPECT, the appropriate statistics is very closely 
Poisson\footnote{{\small For a given scan, the actual distribution of 
the detected raw counts is binomial. However, for acquisition 
times that are shorter than the half life of the radio-isotope, 
the Poisson approximation is very good.}}, 
at least for uncorrected counts. \textit{STIR} includes the following 
utility to find a noise realisation given the mean projection 
data.


Usage:
\cmdline{poisson\_noise [-p {\textbar} --preserve-mean] output\_filename 
{\textbackslash}\\
mean\_projdata\_filename scaling\_factor seed-unsigned-int}


The \textit{scaling\_factor} is used to multiply the input data before 
generating the Poisson random number. This means that a \textit{scaling\_factor} 
larger than 1 will result in less noisy data.


The seed value for the random number generator has to be strictly 
positive. Passing different seeds will result in different noise 
realisations.\\
Without the -p option, the mean of the output data will be equal 
to \textit{scaling\_factor*mean\_of\_input}, otherwise it will be equal 
to \textit{mean\_of\_input}.


The options -p and --preserve-mean are identical.


\subsubsection{
Testing (or using) projectors}

The \textbf{fwdtest} and \textbf{bcktest} programs located in \textbf{STIR/recon\_test/} can 
be used to examine forward or back projectors. They allow to 
project only subregions of the data, but can also do the whole 
projection in one go. 


\subsection{
User-selectable components}
\label{sec:user-selectablecomponents}
The software design of \textit{STIR} has a heavy emphasis on object-oriented 
programming. For the user, the main benefit of this is that it 
is possible to select at \textit{run-time} what particular type of 
e.g. forward projector to use. The following is a list of all 
available components where this type of run-time selection is 
currently available. In the doxygen documentation, look at the 
class documentation for \textbf{RegisteredObject}.


For each component, there are different types, each with its 
own unique name. Each type has its specific set of parameters. 
\\
Each component has its own section, and each type has its subsection. 
The name of the subsection is the unique identifier of the type.


In a parameter file, the selection of the type would look for 
instance as

\begin{verbatim}
This Program Parameters:=
some parameter:=
filter type := my preferred filter type 
my preferred filter type parameters:= ; REQUIRED keyword 
par 1:= 
... 
end my preferred filter type parameters := ; REQUIRED keyword
another program parameter :=
;etc
end :=
\end{verbatim}


Please note that the first and last keywords of a particular 
type \textit{have} to be included in the parameter file, even if no 
additional parameters are given.


\subsubsection{
Available output file formats}
\label{sec:outputfileformats}

\textit{STIR} can write images in a number of different formats. Currently, 
a single 3D image is written per file, \textit{i.e.} no multi-frame 
or gate files yet. The only exceptions to this rule are the \textbf{conv\_to\_ecat6} 
and \textbf{conv\_to\_ecat7} utilities.\\
Currently, the default output file format is Interfile (see below), 
although this could be changed (for most \textit{STIR} programs) by 
editing the file\\
 \textbf{include/stir/IO/DefaultOutputFileFormat.h}\\ 
(not recommended).

{ \subsubsubsection{Common parameters}
}
\label{ref:outputcommonparameters}
The following parameters are common to all file formats. However, 
the implementation for a particular file format could ignore 
the value of these parameters if it does not support it. These 
parameters follow the Interfile 3.3 syntax, \textit{except} that the 
``short float'' and ``long float'' values for ``number format'' 
keyword are not supported. Use ``float'' instead.


\begin{description}
\item[byte order] \texttt{<}string\texttt{>}\linebreak
\raggedright
 values: littleendian {\textbar} bigendian\linebreak
 default: native byte order

\item[number format] \texttt{<}string\texttt{>}\linebreak
\raggedright
 values: bit {\textbar} ascii 
{\textbar} signed integer {\textbar} 
 unsigned integer {\textbar} float;\linebreak
default: float
\item[number of bytes per pixel] \texttt{<}integer\texttt{>}\linebreak
 default: 4
\end{description}


If ``byte order'' is omitted, the default value corresponds to 
the native byte order of the computer the program runs on.\\
Summarising: if none of these keywords is specified, data are 
written as 4-byte floats in the native byte order, unless the 
specific file format has another default.



\textbf{Warning:} Note that only essential information is written 
in the headers. In particular, frame duration etc are currently \textit{not} 
filled in.

{ \subsubsubsection{Interfile}
}
\label{sec:outputinterfile}
The most comprehensively supported file format in the library 
is a newly proposed version of interfile. More details about 
this type can be found on the \textit{STIR} website.
Interfile image files are written as a pair of files
\textit{image\_filename.hv,
image\_filename.v}\\
where \textit{image\_filename.hv} is the header text file and \textit{image\_filename.v} 
is the data file. In addition, we currently write a \textit{.ahv} 
file which uses Interfile 3.3 conventions, with a tweak for the 
\texttt{slice thickness} keyword to work-around an Analyze\texttrademark{} 
bug (see the comments for write\_basic\_interfile\_image\_header() 
in \textbf{STIR}/\textbf{IO/interfile.cxx}). The \textit{.ahv} file is probably 
also readable by other programs capable of reading Interfile 
3.3.


If this output file format is used, and a filename without extension 
is specified for output, or when the filename has an extension \textit{.hv}, 
the above naming conventions hold. If a filename with another 
extension is specified, this name is used for the name of the 
binary file. 


When a file must be specified for reading as a parameter for 
a \textit{STIR} utility or reconstruction program, the name of the \textit{.hv} 
header file should be given. 



\textbf{Warning} The interfile 3.3 standard does not allow to specify 
scale factors for the data. Hence, the \textit{.ahv} file has no scale 
factors. This means that any program that reads the \textit{.ahv} 
file \textbf{will have improperly scaled images}, unless the scale 
factor is 1. However, when using float output, \textit{STIR} automatically 
writes data with scale factor equal to 1, so as long as your 
non-\textit{STIR} program knows about float data, everything will 
be all right. The newly proposed Interfile standard does use 
a scale factor, and the \textit{.hv} file follows this convention. 
However, currently probably only \textit{STIR} programs know about 
this convention.\footnote{{\small The utilities distributed by the UCL, 
Louvain-la-Neuve, Belgium on \R2Lurl{ftp://ftp.topo.ucl.ac.be/pub/ecat }{ftp://ftp.topo.ucl.ac.be/pub/ecat} 
follow a different convention of reading the scale factor. Currently, \textit{STIR} 
writes \textit{.hv} files that can be read correctly by those utilities, 
including the \textit{mediman} image viewer and (x)medcon. Note however 
that \textit{mediman} cannot read float data.}}

{ \subsubsubsubsection{Parameters}
}

This file format currently has no extra parameters, except for 
the start and stop keywords.

\begin{verbatim}
Interfile Output File Format Parameters :=
; any parameters common to all file formats
End Interfile Output File Format Parameters :=
\end{verbatim}

Currently, data is written always in the native byte order, and 
only signed short, unsigned short or float data are supported 
(although this could easily be extended to 1 or 4 byte integers 
or doubles).

{ \subsubsubsection{ECAT6}
}

If this output file format is used, and a filename without extension 
is specified for output, \textit{.img} will be added to the filename.

{ \subsubsubsubsection{Parameters}
}

This file format currently has only 1 extra parameter, except 
for the start and stop keywords.

\begin{verbatim}
ECAT6 Output File Format Parameters :=
; any parameters common to all file formats
default scanner name := <string>
End ECAT6 Output File Format Parameters :=
\end{verbatim}

The scanner name has to be one of the values listed in \textbf{buildblock/Scanner.cxx}, 
but generally follows the format \textit{ECAT 953} (this is the default). 
Currently, the value of this keyword is \textit{always} used, even 
for other images reconstructed from data from other scanners.

Currently, data is written always in the little-endian byte order, 
and only signed short data are supported.

{ \subsubsubsection{ECAT7}
}

The file format is only available when the ECAT7 support is enabled 
during compilation, see section \ref{sec:ECAT67support}.\\
If this output file format is used, and a filename without extension 
is specified for output, \textit{.img} will be added to the filename.

{ \subsubsubsubsection{Parameters}
}

This file format currently has only 1 extra parameter, except 
for the start and stop keywords.

\begin{verbatim}
ECAT7 Output File Format Parameters :=
; any parameters common to all file formats
default scanner name := <string>
End ECAT7 Output File Format Parameters :=
\end{verbatim}

The scanner name has to be one of the values listed in \textbf{buildblock/Scanner.cxx}, 
but generally follows the format \textit{ECAT 962} (this is the default). 
Currently, the value of this keyword is \textit{always} used, even 
for other images reconstructed from data from other scanners.


Currently, data is written always in the big-endian byte order, 
and only signed short data are supported.



\subsubsection{
Available filters or data processors}
\label{sec:filters}
Reconstruction algorithms and some utilities use filters, or in 
general data processing algorithm. In \textit{STIR} 1.x the data-type was restricted
to images, but now other data-types can in principle be used. Some
of the available data processors can work on any data type, but most
only work on images.

The type of data processors 
available to programs is fully extendable (at compile time). 
How to do this is beyond the scope of this document. Here we 
only discuss how to specify a particular data processor at run-time, 
and list the currently available ones. Samples of their parameters 
should be available in \textbf{STIR/samples}.


Each data processor has a unique name associated to it (given 
as the head of its subsection below). This name has to be used 
as value of a filter type keyword (or interactive question). 
Depending on the filter type, different parameters have to be 
given in the.par file (or will be asked interactively).


In addition, it is possible to specify the name \textit{None}. This 
is the default for all keywords that ask for the type of 
the data processor .


See also the online documentation for the class \textbf{DataProcessor}.

{ \subsubsubsection{Separable Convolution}
}

This implements spatial non-periodic convolution with a 3D separable 
filter. The kernel is given in voxel units (\textit{not} in Fourier 
space).\\
This filter applies a 1D convolution in all directions (z,y,x) 
with potentially a different filter kernel for every direction.\\
When parsing, the filter coefficients are read as a list of numbers 
for each direction. The following conventions is used:\\
{\textbullet}\tab 
A list of 0 length (which is the default) corresponds to no filtering.\\
{\textbullet}\tab When the list contains an even number of data, a 0 is 
appended (at the end).\\
{\textbullet}\tab 
After this, the central element of the list corresponds to the 
0-th element in the kernel, see below.


Convolution is non-periodic. In each direction, the following 
is applied: 
\[
\mathrm{out}_i = \sum_{j} \mathrm{kernelforthisdirection}_{j} \mathrm{in}_{i-j}
\]

Note that for most kernels, the above convention means that the 
zero-index of the kernel corresponds to the peak in the kernel. 



Elements of the input array that are outside its index range 
are considered to be 0. 


\textbf{Warning} There is NO check if the kernel coefficients add 
up to 1. This is because not all filters need this (e.g. edge 
enhancing filters).

{ \subsubsubsubsection{Parameters}
}
\begin{verbatim}
Separable Convolution Filter Parameters:=
x-dir filter coefficients:= list_of_numbers
y-dir filter coefficients:= list_of_numbers
z-dir filter coefficients:= list_of_numbers
END Separable Convolution Filter Parameters:=
\end{verbatim}

{ \subsubsubsubsection{Example input for a low-pass filter in x,y, no filtering 
in z}
}
\begin{verbatim}
Separable Convolution Parameters := 
x-dir filter coefficients := {0.25,.5,.25} 
y-dir filter coefficients := {0.25,.5,.25} 
;z-dir filter coefficients :=
END Separable Convolution Parameters :=
\end{verbatim}

{ \subsubsubsection{Separable Cartesian Metz}
}

This is a separable 3D Metz filter for images discretised on 
a Cartesian voxel grid. These filters are composed of 3 1D filter 
kernels, one for the x,y and z directions each. Each 1D filter 
is specified by a Full Width at Half Maximum (FWHM) parameter 
(in millimetres), Metz power parameter and maximum spatial kernel 
width. 



With each triple of parameters, one can associate a 1D continuous 
Metz filter kernel whose frequency response \textit{M(f)} is given 
by the formula

\[
M(f)=
{1-(1-G(f)^2)^{N+1}
\over
G(f)
}
\]

Where \textit{G(f)} is the frequency response (\textit{G(0)} =1) of a zero-mean 
Gaussian distribution function with the associated FWHM and \textit{N} 
is the associated Metz power parameter. Note that in the special 
case \textit{N}=0, the filter reduces to the Gaussian filter \textit{G(f)}. 
If a strictly positive Metz power parameter is chosen, a non-trivial 
Metz filter results whose frequency response possesses an amplifying 
middle frequency band (see also [Jac00]).



The filtering routines apply the filter by computing discretised 
versions of the 1D continuous filter kernels in the space domain 
and convolving with the image in the appropriate direction. The 
following rules apply to the construction and application of 
filters:



(i) The value 0.0 mm is permissible for the filter FWHM parameter. 
Selecting a FWHM of 0.0 mm produces trivial filter kernels, i.e. 
impulse functions, enabling one to disable the filter in any 
direction. 



(ii) In accordance with sampling theory, the routine that constructs 
the 1D filter kernels bandlimits the continuous filter to the 
hypothesized Nyquist frequency of the image (i.e. 1/(2*voxel\_size) 
of the image in the associated direction). Note that this is 
equivalent to convolving with a sinc function in the space domain 
and may yield negative kernel values where unexpected (e.g. in 
a Gaussian filter kernel). In general, when reasonably large 
FWHMs are selected (a few multiples of the voxel dimension), 
the bandlimiting has no effect on the filter construction.



(iii) The spatial kernel width can be limited, which will set 
any other values to 0. If this is not done, the kernel width 
is determined by the first value which is smaller in absolute 
value than 10{\textasciicircum}-6 times the central kernel value.



(iv) Filter kernels computed by the software may contain non-positive 
values. This is partly because, in general, a 1D continuous Metz 
filter response function may contain non-positive value and so 
also will its samples. In addition, non-positive kernel values 
may arise due to the bandlimiting operations described in (ii). 



Consequently, filtered positive images may likewise contain non-positive 
values. 

{ \subsubsubsubsection{Parameters}
}

Sample parameters are given below

\begin{verbatim}
separable cartesian metz filter parameters :=
x-dir filter fwhm (in mm) := 6
y-dir filter fwhm (in mm) := 6
z-dir filter fwhm (in mm) := 6
x-dir filter metz power := 2
y-dir filter metz power := 2
z-dir filter metz power := 2
x-dir maximum kernel size := 129
y-dir maximum kernel size := 129
z-dir maximum kernel size := 31
end separable cartesian metz filter parameters := 
\end{verbatim}


An explanation of these parameters is given here for the x-direction 
(others are obvious extensions)
\begin{description}
\item[x-dir filter FWHM (in mm)]
The Full Width at Half Maximum of the Gaussian filter kernel 
from which the \textit{x} direction Metz filter kernel are derived.

\item[x-dir filter Metz power]
The exponent parameter for the \textit{x} direction Metz filter kernel.

\item[x-dir maximum kernel size]
The maximum width of the kernel (in pixels). Prior to version 
0.93 of the PARAPET library this was fixed to the number of pixels 
in the input image (in that direction).
\end{description}

{ \subsubsubsection{Median}
}
\label{sec:median}
This applies a straightforward 3D (or 2D) median filter on the 
image.

{ \subsubsubsubsection{Parameters}
}

Sample parameters are given below

\begin{verbatim}
Median Filter Parameters := 
mask radius x := 1  
mask radius y := 1 
mask radius z := 1 
End Median Filter Parameters:=
\end{verbatim}


A radius of 0 means no filtering in that direction, a radius 
of 1 means the median will be computed over a mask of 3 pixels, 
and so on.


{ \subsubsubsection{Truncate To Cylindrical FOV}
}
This image processor will set all voxels to 0 outside a certain radius.
TODO this needs updating.
{ \subsubsubsubsection{Parameters}
}

Sample parameters are given below

\begin{verbatim}
Truncate To Cylindrical FOV Parameters:=
; default use x^2 + y^2 < R^2
; if set to 0, will use <=
strictly_less_than_radius:=1 
End Truncate To Cylindrical FOV Parameters:=
\end{verbatim}

{ \subsubsubsection{Threshold Min To Small Positive Value}
}
This is a generic data processor that will work on any type of data.

Since strict positivity is a preferred property of images in 
many circumstances, a post-thresholding of the filtered data 
is sometimes applied to truncate non-positive values. The thresholding 
rule currently used is:\\
(1)\tab 
If the entire filtered data is non-positive, the data is uniformly 
set to a hard-coded strictly positive parameter SMALL\_NUM\texttt{<}\texttt{<}1.\\
(2)\tab 
Otherwise, all non-positive entries in the data are set to SMALL\_NUM 
times the minimum strictly positive value.


{ \subsubsubsection{Chained Data Processor}
}
This is a generic data processor that will work on any type of data.

This data processor allows subsequent application of 2 other 
data processors on the image. It can for example be used to 
first Metz filter an image and then threshold it.

{ \subsubsubsubsection{Parameters}
}

Example parameters are given by

\begin{verbatim}
Chained Data Processor Parameters :=
Data Processor to apply first := None
Data Processor to apply second := None
END Chained Data Processor Parameters :=
\end{verbatim}

Obviously, in normal practice the Data Processor keywords will 
have values given by any of the listed data processors in this 
section (including Chained Data Processor again).



\subsubsection{
Incorporating prior information}
\label{sec:priors}
Some iterative reconstruction algorithms allow the incorporation 
of \textit{a priori} information, for instance the One Step Late algorithm 
(implemented as OSMAPOSL). At the moment, 'generalised' priors 
are used, where we mean that we need to know only the gradient 
of the actual (log of the) prior function.\\
Which priors are available to programs is fully extendable (at 
compile time). How to do this is beyond the scope of this document. 
Here we only discuss how to specify a particular prior, and list 
the currently available ones. Samples of their parameters should 
be available in \textbf{STIR/samples}.



Each prior has a unique name associated to it (given as the head 
of its subsection below). This name has to be used as value of 
a prior type keyword (or interactive question). Depending on 
the prior type, different parameters have to be given in the.par 
file (or will be asked interactively).


In addition, it is possible to specify the name \textit{None}. This 
is the default for all prior keywords.


See also the online documentation for the class \textbf{GeneralisedPrior}.


{ \subsubsubsubsection{Parameters}
}

These are the keywords that can be used for all priors.
  \begin{verbatim}
   penalisation factor := <float>
  \end{verbatim}
  \noindent where the \texttt{penalisation factor} is usually called
  $\beta$ in the literature, and is just a global scale factor for the
  prior.

{ \subsubsubsection{FilterRootPrior}
}

This prior is an extension of the idea first developed for the 
Median Root Prior [Ale97]. The prior takes any Data Processor 
(i.e. a filter), and computes the prior 'gradient' as 
\[
G_{v\,} =\lambda _{v} /F_{v}  -1
\]
\noindent where 
$\lambda _{v} $
 is the image where to compute the gradient, and 
$F_{v} $ is the image obtained by filtering $\lambda $.

Note that for nearly all filters, this is not really a prior, as 
this 'gradient' is \textit{not} the gradient of a function. This can 
be checked by computing the 'Hessian' (i.e. the partial derivatives 
of the components of the gradient). For most (interesting) filters, 
the Hessian will not be symmetric.


The Median Root Prior is obtained by using Median (see \ref{sec:median}) 
as Data Processor.

{ \subsubsubsubsection{Parameters}
}

These are the keywords that can be used in addition to the ones listed in \ref{sec:priors}..
  \begin{verbatim}
  FilterRootPrior Parameters :=
  penalisation factor := 1
  ; you can use any data processor here
  ; the next parameters specify a 3x3x3 median
  Filter type := Median
    Median Filter Parameters :=
    mask radius x := 1   
    mask radius y := 1
    mask radius z := 1
    End Median Filter Parameters:=

  END FilterRootPrior Parameters :=
 \end{verbatim}

{ \subsubsubsection{Quadratic}
}

This implements a quadratic Gibbs prior. The gradient of the prior is computed as follows:
  
 \[
  g_r = \sum_dr w_{dr} (\lambda_r - \lambda_{r+dr}) * \kappa_r * \kappa_{r+dr}
  \]
  \noindent where $\lambda$ is the image where the gradient is computed
   and $r$ and $dr$ are indices and the sum
  is over the neighbourhood where the weights $w_{dr}$ are non-zero.

  The $\kappa$ image can be used to have spatially-varying penalties such as in 
  Jeff Fessler's papers. It should have identical dimensions to the image for which the
  penalty is computed. If $\kappa$ is not set, this class will effectively
  use 1 for all $\kappa$'s.

  By default, a 3x3 or 3x3x3 neigbourhood is used where the weights are set to 
  x-voxel\_size divided by the Euclidean distance between the points.
 
{ \subsubsubsubsection{Parameters}
}
  These are the keywords that can be used in addition to the ones listed in \ref{sec:priors}..
  \begin{verbatim}
  Quadratic Prior Parameters:=
  ; next defaults to 0, set to 1 for 2D inverse Euclidean weights, 0 for 3D 
  only 2D:= 0
  ; next can be used to set weights explicitly. Needs to be a 3D array (of floats).
  ' value of only_2D is ignored
  ; following example uses 2D 'nearest neighbour' penalty
  ; weights:={{{0,1,0},{1,0,1},{0,1,0}}}
  ; use next parameter to specify an image with penalisation factors (a la Fessler)
  ; see class documentation for more info
  ; kappa filename:=
  ; use next parameter to get gradient images at every subiteration
  ; see class documentation
  gradient filename prefix:= 
  END Quadratic Prior Parameters:=
  \end{verbatim}

\subsubsection{
Selecting different projector pairs}
\label{sec:projectorpairs}
Many algorithms use both a forward and a back projector. The 
first step to select which one will be used is to say what kind 
of projector pair you want to use, of which there are only 2 
candidates as given below. 


Each projector pair has a unique name associated to it (given 
as the head of its subsection below). This name has to be used 
as value of a 'projector pair type' keyword (or interactive question). 
Depending on the projector pair type, different parameters have 
to be given in the.par file (or will be asked interactively).


Samples of OSMAPOSL reconstruction parameter files selecting 
different types of projectors should be available in \textbf{STIR/samples}.\\
\textbf{Warning:} for most iterative algorithms, it is recommended 
to use matching forward and back-projector. This is unfortunately 
not the default due to historical reasons.

{ \subsubsubsection{Matrix}
}
\label{sec:projectorpairusingmatrix}
Both projectors are based on a single projection matrix, as given 
in Section \ref{sec:projmatrix}.

{ \subsubsubsubsection{Parameters}
}
\begin{verbatim}
Projector Pair Using Matrix Parameters:=
Matrix type := some value
; parameters relevant to this type of matrix
End Projector Pair Using Matrix Parameters:=
\end{verbatim}

See Section \ref{sec:projmatrix} for possible values for the 'matrix type' keyword.

{ \subsubsubsection{Separate Projectors }
}

Forward and back projectors are completely independent of each 
other. In some programs this is necessary to handle images or 
projection data of different sizes. \\
Even if the projectors are used that both use the same type of 
projection matrix, that matrix will not share the same cache 
or memory.

{ \subsubsubsubsection{Parameters}
}
\begin{verbatim}
Projector Pair Using Separate Projectors Parameters:=
Forward projector type:= some value
Back projector type:= some value
End Projector Pair Using Separate Projectors Parameters:=
\end{verbatim}


See Section \ref{sec:forwardprojectors} for possible values for the 'forward projector 
type' keyword and Section \ref{sec:backprojectors} for the 'back projector type' 
keyword.


\subsubsection{
Selecting a forward projector}
\label{sec:forwardprojectors}
It is possible to select the forward projector used at run-time, 
and extend the available ones at compile time. The mechanism 
is exactly the same as for the ImageProcessor hierarchy.


Each projector has a unique name associated to it (given as the 
head of its subsection below). This name has to be used as value 
of a 'forward projector type' keyword (or interactive question). 
Depending on the projector type, different parameters have to 
be given in the.par file (or will be asked interactively).

{ \subsubsubsection{Matrix}
}

This forward projector uses a projection matrix to compute its 
result.

{ \subsubsubsubsection{Parameters}
}
\begin{verbatim}
Forward Projector Using Matrix Parameters:=
Matrix type := some value
End Forward Projector Pair Using Matrix Parameters:=
\end{verbatim}

See Section \ref{sec:projmatrix} for possible values for the 'matrix type' keyword.

{ \subsubsubsection{Ray Tracing}
}

This forward projector uses an optimisation of Siddon's algorithm 
to compute its result. That is, it uses Length of Intersection. 
As it avoids storing the matrix elements, it is currently faster 
than using a forward projector using a ray tracing matrix (see 
Section \ref{sec:projmatrixusingraytracing}). The result is identical though (up to rounding 
errors and possibly the voxels at the edge of the FOV).


See also online documentation for class ForwardProjectorUsingRayTracing.

{ \subsubsubsubsection{Parameters}
}
\begin{verbatim}
Forward Projector Using Ray Tracing Parameters:=
End Forward Projector Using Ray Tracing Parameters:=
\end{verbatim}

This projector currently has no user-selectable parameters. Nevertheless, 
the 2 keywords given above have to follow the 'forward projector 
type' keyword in a parameter file.


\subsubsection{
Selecting a back projector}
\label{sec:backprojectors}
It is possible to select the type of back projector. The mechanism 
is exactly the same as for the ForwardProjector hierarchy.

{ \subsubsubsection{Matrix}
}

This back projector uses a projection matrix to compute its result.

{ \subsubsubsubsection{Parameters}
}
\begin{verbatim}
Back Projector Using Matrix Parameters:=
Matrix type := some value
Back Forward Projector Pair Using Matrix Parameters:=
\end{verbatim}

See Section \ref{sec:projmatrix} for possible values for the 'matrix type' keyword.

{ \subsubsubsection{Interpolation}
}

This back projector uses incremental (piecewise)-linear interpolation 
to compute its result. See [Egg98] for details. It can \textit{only} handle arc-corrected data.


\textbf{Warning:} The current implementation has problems (noticeable 
sometimes at 45 and 135 degrees but mostly at the centre of the image) on Sun 
Sparc, HP and some Opteron processors (for certain optimisation settings
of the C++ compiler).

{ \subsubsubsubsection{Parameters}
}
\begin{verbatim}
Back Projector Using Interpolation Parameters:=
Use piecewise linear interpolation := 1
Use exact Jacobian := 1
End Back Projector Using Interpolation Parameters:=
\end{verbatim}

The 'piecewise linear' keyword allows the user to choose between 
ordinary linear interpolation or piecewise linear interpolation 
[Thi99] in axial direction. The latter approximates Volume of 
Intersection for axially uncompressed data. The piecewise linear 
interpolation is only used when the axial voxel size is half 
the axial\_sampling of the projection data (for the segment in 
question), otherwise linear interpolation is used anyway.


The 'exact Jacobian' keyword selects if the exact or an approximate 
version of the Jacobian is used. The approximation consists in 
taking the value for the central tangential position and is often 
used in analytic algorithms in the literature.


Even when all default values are used, the start and end keywords 
given above have to follow the 'back projector type' keyword 
in a parameter file.


See also online documentation for class BackProjectorByBinUsingInterpolation.



\subsubsection{
Selecting a projection matrix}
\label{sec:projmatrix}
It is possible to select these independently at run-time, and 
extend the available ones at compile time. The mechanism is exactly 
the same as for the ForwardProjector hierarchy.

{ \subsubsubsection{Common parameters to all projection matrices}
}
\label{sec:projmatrixcommon}
The following parameters can be used for all types of projection 
matrices. There default values are indicated below.

\begin{verbatim}
disable caching:= 0
store only basic bins in cache:=1
\end{verbatim}

Here is an explanation of these parameters.

\begin{description}
\item[disable caching] [0,1,0{]}
Normally the elements of the matrix are stored in memory (or 
at least parts of it, see next keyword) after the first use. 
This can be disabled, but this should normally only be done if 
not enough RAM memory is available such that heavy swapping occurs. 
The current procedure is an all-or-nothing cache. In the future, 
it might become possible that an upper memory limit can be given.


\item[store only basic bins in cache] [0,1,1{]}
Most projectors use symmetries to reduce the number of elements 
that need to be computed. For example, if the ring spacing of 
the scanner is an integer multiple of the voxel size in z-direction, 
it is only necessary to compute the matrix elements only for 
1 ring(pair). When caching is enabled, by default only the independent 
elements are cached. If you have plenty of RAM memory, you can 
store all (non-zero) elements. If your system does not start 
swapping, this will speed-up the computation.
\end{description}

{ \subsubsubsection{Ray Tracing}
}
\label{sec:projmatrixusingraytracing}
This projection matrix uses an optimisation of Siddon's algorithm 
to compute its result. That is, it uses Length of Intersection.\\
Currently, the LOIs are divided by voxel\_size.x(), unless NEWSCALE 
is \#defined during compilation time of ProjMatrixByBinUsingRayTracing.cxx. 

It is possible to use multiple LORs in tangential direction. 
The result will then be the average of the various contributions. 
Currently all these LORs are parallel. For a very high number 
of LORs, the result approximates a strip integral (in tangential 
direction).

If the z voxel size is exactly twice the sampling in axial direction, 
2 or 3 LORs are used, to avoid missing voxels. 

It is possible to use a cylindrical or cuboid FOV (in the latter 
case it is going to be square in transaxial direction). In both 
cases, the FOV is slightly 'inside' the image (i.e. it is about 
1 voxel at each side smaller than the maximum possible).

It is possible to reduce the number of symmetries used by this 
matrix. This might be useful if you have plenty of RAM and want 
to speed up the calculations. See also the discussion on caching 
in section \ref{sec:projmatrixcommon}.

\textbf{Warning} Care should be taken to select the number of rays 
in tangential direction such that the sampling is not greater 
than the x,y voxel sizes.

\textbf{Warning} The current implementation assumes that z voxel size 
is either smaller than or exactly twice the sampling in axial 
direction of the segments.

{ \subsubsubsubsection{Parameters}
}

The following parameters can be set (default values are indicated):

\begin{verbatim}
Ray Tracing Matrix Parameters := 
; any parameters appropriate for all matrices
restrict to cylindrical FOV := 1

number of rays in tangential direction to trace for each bin := 1

do symmetry 90degrees min phi := 1 
do symmetry 180degrees min phi := 1
End Ray Tracing Matrix Parameters :=
\end{verbatim}

For the azimuthal angle $\phi$, the following angles are symmetry 
related for a square grid: \\
$\{\phi, 180^{o}-\phi, 90^{o}-\phi, 90^{o}+\phi\}$.\\
Two boolean parameters allow to select which angles should be 
considered as related:
\begin{description}
\item[all 4]
(do\_symmetry\_90degrees\_min\_phi=true)
\item[only $\{ \phi, 180- \phi \}$]
(do\_symmetry\_90degrees\_min\_phi=false,  \\
do\_symmetry\_180degrees\_min\_phi = true)
\item[none]
(do\_symmetry\_90degrees\_min\_phi=false, \\
do\_symmetry\_180degrees\_min\_phi = false)
\end{description}

Note that when \texttt{do\_symmetry\_90degrees\_min\_phi=true}, 
it is irrelevant what the value is 
of \texttt{do\_symmetry\_180degrees\_min\_phi}. This is because 
otherwise a non-consecutive range in $\phi$ would have to be used.


The symmetry in $\phi$ is automatically reduced for non-square grids 
or when the number of views is not a multiple of 4.



In addition, there is a keyword \texttt{use actual detector boundaries}. 
However, changing its default value is currently not recommended. 




Note that even when all default values are used, the start and 
stop keywords given above have to follow the 'matrix type' keyword 
in a parameter file.


See also online documentation for class ProjMatrixByBinUsingRayTracing. 

{ \subsubsubsection{From File}
}
TODO
\label{sec:projmatrixfromfile}

{ \subsubsubsubsection{Parameters}
}

The following parameters can be set (default values are indicated):

\begin{verbatim}

\end{verbatim}


\subsubsection{
Selecting a bin normalisation procedure}
\label{sec:binnormalisation}
In PET, a procedure called 'normalisation' is used which is essentially 
a calibration procedure for every detector pair. It provides 
a multiplicative factor for every bin, or element of the projection 
data.\\
The library can provide different types of normalisation procedures. 
It is possible to select these independently at run-time, and 
extend the available ones at compile time. The mechanism is exactly 
the same as for the ForwardProjector hierarchy.


In addition to the types listed below, you can also enter 'None', 
which means that the data won't normalised at all.

{ \subsubsubsection{From Projdata}
}

This can be used when the normalisation factors are stored simply 
as projection data. Currently, these data have to have exactly 
the same characteristics (size etc.) as the projection data which 
are going to be normalised. Note that the stored factors have 
to be the ones you'd apply to normalise the data (and not their 
reciprocal).

{ \subsubsubsubsection{Parameters}
}
\begin{verbatim}
Bin Normalisation From ProjData :=
normalisation projdata filename:= norm.hs
End Bin Normalisation From ProjData:=
\end{verbatim}

See also online documentation for class BinNormalisationFromProjData.

{ \subsubsubsection{From ECAT7}
}

This can be used when normalising ECAT7 data. CTI/Siemens stores
the normalisation data in a files normally ending on
\texttt{.n} or \texttt{.N}. Dead-time correction is also
supported, although awkwardly. To get dead-time correction
to work, you need to specify the singles rates for the 
scan\footnote{In future, hopefully this will not be necessary.
This work-around is needed because \textit{STIR} currently does not
directly read any of the meta-data in the headers of sinograms etc.}
{ \subsubsubsubsection{Parameters}
}
\begin{verbatim}
Bin Normalisation From ECAT7:=
  normalisation filename:= STUDY.n
  singles rates := Singles From ECAT7 File
  Singles Rates From ECAT7 File:=
      ECAT7_filename := ecat7_sinogram.S
  End Singles Rates From ECAT7:=
End Bin Normalisation From ECAT7:=
\end{verbatim}

In addition, you currently have to set the time frame information in the .par file.

See also online documentation for class BinNormalisationFromECAT7.

{ \subsubsubsection{From Attenuation Image}
}
\label{sec:binnormalisationfromattenuationimage}
This can be used for attenuation correction factors (ACFs) if 
you do not have the ACFs but an attenuation image (or mu-map). 
The ACFs are found by forward projecting the attenuation image, 
multiplying the result with -1, and exponentiating, i.e. using 
Beer's law.



\textbf{Warning} Attenuation image data are supposed to be in units \textit{cm}$^{-1}$. 
(Reference: water has $\mu=.096 \mathit{cm}^{-1}$.)

{ \subsubsubsubsection{Parameters}
}
\begin{verbatim}
Bin Normalisation From Attenuation Image:=
attenuation_image_filename := <string>
forward projector type := <string>
End Bin Normalisation From Attenuation Image :=
\end{verbatim}


Default forward projector is ForwardProjectorByBinUsingRayTracing 
(see section \ref{sec:forwardprojectors}).


See also online documentation for class BinNormalisationFromAttenuationImage.

{ \subsubsubsection{Chained }
}
\label{sec:chainedbinnormalisation}
This can be used to apply two normalisation one after the other. 
For example, a first one could be the 'usual' normalisation factor, 
while a second one could be the attenuation factors.

{ \subsubsubsubsection{Parameters}
}
\begin{verbatim}
Chained Bin Normalisation Parameters:=
Bin Normalisation to apply first:= some_bin_normalisation_type
; parameters for this type

Bin Normalisation to apply second:= some_bin_normalisation_type
; parameters for this type

END Chained Bin Normalisation Parameters:=
\end{verbatim}

See also online documentation for class ChainedBinNormalisation.



\subsection{
Display}

Some of the programs (e.g. in utilities/, recon\_tests/) use 
the display routines. Which version is actually used depends 
on the compilation settings (in particular the \textit{GRAPHICS} make 
flag), see section \ref{sec:compilationgraphics}. 


\subsubsection{
X Windows display}
\label{sec:display}
This provides a (very basic) way of displaying bitmaps when using 
X windows. This works by creating a new window where some of 
the bitmaps are displayed. To proceed, you have to make this 
window the 'focus' (how to do this depends on your window manager 
but usually you have to move the cursor over it, or click the 
title bar) and then press any key. In your original terminal 
window you will then be asked if you want to continue with the 
next set of bitmaps until no more are left.


In addition, when your X server supports the Pseudocolor visual, 
you can cycle between 4 different colour scales by pressing Mouse 
button 2 while the 'bitmap' window is selected.


\textbf{Note:} When you are using cygwin on Windows with the XFree86 
X server, there is currently a somewhat strange problem. If you 
forget to switch focus to the 'bitmap' window and press a key 
(which will end up in the terminal window you started the program 
in), you might not be able to continue the display in the usual 
way and have to abort the program by pressing Ctrl-C. 


\textbf{Warning} In the current implementation, it seems to happen 
occasionally that not all colours in the colour scale have been 
allocated properly. This can be seen by looking at the side bar 
displaying the colour scale. Unfortunately, this effect might 
give you wrongly coloured regions (usually spots) in the image.


\subsubsection{
PGM display}

This 'display' mode actually writes out a file in the Portable 
Greyscale Map format, which can be read by various graphics programs 
(like Paint Shop Pro or xv).


\subsubsection{
MathLink display}

This mode is (somewhat) useful if you have \textit{Mathematica\texttrademark}, 
and want to pipe the data into \textit{Mathematica}. (It is probably 
easier to read the binary data from file though using the \textit{Mathematica} command \texttt{BinaryRead}). 
Sample \textit{Mathematica} statements:

\begin{verbatim}
(* create link before starting the (first) display in your STIR 
program*)

link=LinkCreate[``STIR''];

(* read data from link *)
data3d=LinkRead[link];


(* display the 3rd image *)
ListPlot3D[data3d[[3]], Mesh->}False];
(* read data from link from next display *)
nextdata3d=LinkRead[link];
(* and so on *)

(* close link at end of STIR program *)
LinkClose[link];
\end{verbatim}


\section{B-spline interpolation in STIR}

\label{sec:BSplines}
Special classes to interpolate data using B-Splines [Uns99] have been implemented in STIR. 
The B-Splines coefficients are estimated with a fast method that incorporates FIR and IIR 
filters [Uns93]. The level of B-splines that are currently available are 0, 1 (linear), 
2 (quadratic), 3 (cubic), 4 (quartic) and 5 (quintic). Moreover, a synthetic spline has also been 
implemented based on B-splines for maximum order interpolation but minimal support, i.e. MOMS 
splines [Blu01]. In all cases no regularization has been incorporated during the interpolation 
procedures, therefore special care should be taken when higher than the first order B-splines 
are used, as noise may be enhanced at the interpolated data. Interpolation can be also performed 
in sinogram space. However this space is regular only in axial, and not in radial, angular and 
azimuthal directions. Therefore, the current interpolation can be cubic only in axial direction 
while for the rest directions a choice of linear interpolator is safer. In future with some 
additional classes, reconstruction could be performed to obtain B-spline coefficients directly 
from the projection sinograms, which can be useful for better noise properties [Nic02].

\section{
Directories in the STIR tree}
\textit{STIR} can roughly be split into a library applications and applications.
\footnote{The distinction is not complete as some applications are implemented
in terms of classes that actually end up in the library.}

See the doxygen documentation for an overview of the directory tree with brief descriptions.

\section{
Future developments and Support}

The \textit{STIR} library, in its current state, possesses many capabilities. 
The developers, however, look forward to still further increases 
in the flexibility and power of the software. Some of the developments 
being discussed are:

\begin{itemize}
\item
expanded library of polymorphic classes (e.g. image grids, and 
ordered subsets)
\item
additional scanners and modalities such as SPECT
\item
additional data formats support, without conversion to Interfile.
\item
point-spread function reconstruction
\end{itemize}


While support for the library is on a voluntary basis, users 
of the library are encouraged to subscribe to relevant \textit{STIR} 
mailing lists (see the `Mailing Lists' section of the \textit{STIR} website 
\R2Lurl{http://stir.sourceforge.net/ }{http://stir.sourceforge.net}) 
where they can follow developments of the software and obtain 
helpful information from other users. Questions will ONLY be 
answered (if at all) when directed to the mailing list.



Below, we list of some of the features that might make it into the next releases. 
However, which feature is actually finalised/implemented depends 
on the needs of the developers. If you want one of these features 
and are willing to help, let us know.
\begin{itemize}
\item
More automatic testing programs
\item
More algorithms: potentially ART, FORE [Def95], 
OSCB [Ben99b] 
\item
More projectors
\item
More priors
\item
Extending the parallelisation of OSMAPOSL and OSSPS to FBP3DRP etc, or using OPEMMP
\item
Compatibility of the interpolating backprojector with recent 
processors.
\item
More kinetic models: Spectral Analysis, Logan Plot
\end{itemize}





\section{
References}

{[}Ahn2003] S. Ahn and J. A. Fessler, \textit{Globally convergent image reconstruction for
emission tomography using relaxed ordered subsets algorithms,} IEEE
Trans. Med. Imag., vol. 22, no. 3, pp. 613-626, May 2003.

{[}Ale97] Alenius S and Ruotsalainen U \textbf{(1997)} Bayesian image 
reconstruction for emission tomography based on median root prior. \textit{European 
Journal of Nuclear Medicine} , Vol. 24 No. 3: 258-265.

{[}Ang11] Angelis G I, Thielemans K, Tziortzi A C, Turkheimer F E and Tsoumpas C  
\textbf{(2011)}. Convergence Optimization of Parametric MLEM Reconstruction for Estimation of Patlak Plot Parameters.  
\textit{(Computerized Medical Imaging and Graphics}, 35:407-416,  
\textit{doi:10.1016/j.compmedimag.2011.01.002}

{[}Ben99] Ben-Tal A, Margalit T and Nemirovski A \textbf{(1999)} The 
ordered subsets mirror descent optimization method with application 
to tomography. \textit{Research report \#2/99, March 1999, MINERVA 
Optimization Center, Faculty of Industrial Engineering and Management, 
Technion -- Israel Institute of Technology}.

{[}Ben99b] Ben-Tal A and Nemirovski A \textbf{(1999)} The conjugate 
barrier method for non-smooth, convex optimization. \textit{Research 
report \#5/99, October 1999, MINERVA Optimization Center, Faculty 
of Industrial Engineering and Management, Technion -- Israel Institute 
of Technology.}

[Bei08] Beisel T, Lietsch S, and Thielemans K \textbf{(2008)}
A Method for OSEM PET Reconstruction on Parallel Architectures Using STIR.
\textit{Proc. of the IEEE Medical Imaging Conference, Dresden, Germany, 2008}.

{[}Blu01] Blu T, Thcvenaz P, Unser M \textbf{(2001)}. MOMS: maximal-order interpolation of minimal support.  \textit{IEEE Trans Image Processing} 10(7): 1069-1080.

{[}Bou97] Boudraa \textsc{A E O} \textbf{(1997)} Automated detection of the 
left ventricular region in magnetic resonance images by Fuzzy 
C-Means model. \textit{Int J of Cardiac Imag} ; 13: 347-355.

{[}Dau86] Daube-Witherspoon M E and Muehllener G \textbf{(1986)} An 
iterative space reconstruction algorithm suitable for volume 
ECT. \textit{IEEE Trans. Med. Imaging}, vol. MI-5: 61-66.

{[}Dau97] Daube-Witherspoon M E and Muehllehner G \textbf{(1987)} 
Treatment of axial data in three-dimensional PET, 
J. Nucl. Med. 28, 171-1724.

{[}Def95] Defrise M. \textbf{(1995)} ``A factorization method for the 
3D X-ray transform'' \textit{Inverse Problems,} \textbf{11} pp. 983-994.


{[}Def97] Defrise M, Kinahan P E, Townsend D W, Michel C, Sibomana 
M and Newport D F \textbf{(1997)} Exact and approximate rebinning 
algorithms for 3-D PET data. \textit{IEEE Trans. Med. Imaging}, MI-16: 
145-158.

{[}Egg98] Egger M L, Joseph C and Morel C \textbf{(1998)} Incremental 
beamwise backprojection using geometrical symmetries for 3D PET 
reconstruction in a cylindrical scanner geometry. \textit{Phys Med. 
Biol.}, 43: 3009-3024.


{[}Gem84] Geman S and Geman D \textbf{(1984)} Stochastic relaxation, 
Gibbs distributions, and the Bayesian restoration of images. \textit{IEEE 
Trans PAMI}, 6: 721-741.

{[}Gem85] Geman S and McClure D \textbf{(1985)} Bayesian image analysis: 
an application to single photon emission tomography. \textit{in Proc. 
American Statistical Society, Statistical Computing Section (Washington, 
DC)} 12-18.


 {[}Gre90] Green P J \textbf{(1990)} Bayesian reconstruction from emission 
tomography data using a modified EM algorithm. \textit{IEEE Trans. 
Med. Imaging}, MI-9: 84-93.


{[}Heb89] Hebert T J and Leahy R M \textbf{(1989)} A generalized EM 
algorithm for 3-D Bayesian reconstruction from Poisson data using 
Gibbs priors. \textit{IEEE Trans. Med. Imaging}, MI-8: 194-202.


{[}Her80] Herman G T \textbf{(1980)} Image Reconstruction from Projections: 
The fundamentals of Computational Tomography. \textit{Academic Press, 
New York}.

{[}Hud94] Hudson H M and Larkin R S \textbf{(1994)} Accelerated image 
reconstruction using ordered subsets of projection data. \textit{IEEE 
Trans. Med. Imaging}, MI-13: 601-609.


{[}Jac00] Jacobson M, Levkovitz R, Ben-Tal A, Thielemans K, Spinks 
T, Belluzzo D, Pagani E, Bettinardi V, Gilardi M C, Zverovich 
A and Mitra G \textbf{(2000)} Enhanced 3D PET OSEM Reconstruction 
using inter-update Metz filters. \textit{Phys. Med. Biol}. \textbf{45} No.8 
(2000) 2417-2439\textit{l.}


{[}Kin89] Kinahan P E and Rogers J G \textbf{(1989)} Analytic 3D image 
reconstruction using all detected events. \textit{IEEE Trans. Nucl. 
Sci.}, 36: 964-968.


{[}Lab97] Labbe C, Ashburner J, Koepp M, Spinks T, Richardson M 
and Cunningham V \textbf{(1997)} Accurate PET quantification using 
correction for partial volume effects within cerebral structures. \textit{Neuroimage}, 
5: B12.


{[}Lab99a] Labb\'{e} C, Thielemans K, Belluzzo D, Bettinardi V, Gilardi 
MC, Hague DS, Jacobson MW, Kaiser S, Levkovitz R, Margalit T, 
Mitra G, Morel C, Spinks T, Valente P, Zaidi H and Zverovich A\textit{: 
An object-Oriented Library for 3D PET Reconstruction using Parallel 
Computing}, Proceedings of Bildverarbeitung fuer die Medizin 
1999, Algorithmen-Systeme-Anwendungen, \textit{Informatik} 
\textit{aktuell, Springer}, Eds. H. Evers, G. Glombitza, T. Lehmann, 
H.-P. Meinzer, pp 268-272. 

{[}Lab99b] Labb\'{e} C, Thielemans K, Zaidi H and Morel C: An object-oriented 
library incorporating efficient projection/backprojection operators 
for volume reconstruction in 3D PET\textit{. Proc. of 3D99 Conference}, 
June 1999, Egmond aan Zee, The Netherlands 


{[}Lal93] Lalush D S and Tsui M W \textbf{(1993)} A general Gibbs prior 
for Maximum a posteriori reconstruction in SPET. \textit{Phys. Med. 
Biol.}, 38: 729-741.

{[}Lan90] Lang K \textbf{(1990)} Convergence of EM Image reconstruction 
algorithms with Gibbs smoothing. \textit{IEEE Trans. Med. Imaging}, 
MI-9: 4.


{[}Man97] Manders Jones H \textbf{(1997)} A Computational Investigation 
of the solution of large scale Optimization problems. \textit{Ph.D 
Thesis, Department of Mathematics and Statistics. Brunel, The 
University of West London. Oct 1997}.


{[}Mar99] Margalit T, Gordon E, Jacobson M, Ben-Tal A, Nemirovski 
A and Levkovitz R \textbf{(1999)} The ordered sets mirror descent and 
conjugate barrier optimization algorithms adapted to the 3D PET 
reconstruction problem. Technion internal report and PhD thesis of T. Margalit

{[}Mus01] Mustafovic S and Thielemans K, \textbf{(2001)} Additive and Multiplicative 
versions of the Maximum A Posteriori Algorithm with the Median 
Root Prior. poster at \textit{IEEE Med. Img. Conf. 2001.}

{[}Mus04] Mustafovic S and Thielemans K, \textbf{(200}\textbf{4}\textbf{)} \textit{Object 
Dependency of Resolution in Reconstruction Algorithms with Inter-Iteration 
Filtering Applied to PET} , IEEE Trans. Med. Im. 23 (4): (2004) 
433-446.

{[}Mus02] Mustafovic S and Thielemans K, \textit{,} \textbf{(2002)}  Comparison 
of Unconventional Inter-Filtering Methods to Penalised-likelihood 
for Space-invariant Tomographs, poster at \textit{IEEE Medical Imaging 
Conf. 2002}.

{[}Nem78] Nemirovski A and Yudin D \textbf{(1978)} Problem complexity 
and method efficiency in optimization. \textit{Nauka Publishers, 
Moscow, 1978 (in Russian); English translation: John Wiley \& 
Sons, 1983}.

{[}Nic02] Nichols, T. E., J. Qi, E. Asma and R. M. Leahy \textbf{(2002)}. Spatiotemporal reconstruction of list-mode PET data.  \textit{IEEE Trans Med Imaging} 21(4): 396-404.

{[}Pat83] Patlak C S, Blasberg R G, Fenstermacher J D \textbf{(1985)}  Graphical evaluation of blood-to-brain 
transfer constants from multiple-time uptake data. \textit{J Cereb Blood Flow Metab} 3(1): p. 1-7.

{[}Pat85] Patlak C S, Blasberg R G \textbf{(1985)}  Experimental and Graphical evaluation of blood-to-brain 
transfer constant from multiple-time uptake data: Generalizations. \textit{J Cereb Blood Flow Metab} 5: p. 584-90.

{[}Poe03{]} Poenisch F, Enghardt W, and Lauckner K \textbf{(2003)}, 
Attenuation and scatter correction for in-beam positron emission tomography 
monitoring of tumour irradiations with heavy ions. \textit { Phys Med Biol } 48(15): 2419-2436. 

 {[}PAR1.3] Levkovitz R, Zibulevsky M, Labbe C, Zaidi H and Morel 
C \textbf{(1997)} Determination of a Set of Existing Algorithms for 
Comparative Evaluation. \textit{PARAPET documentation for D1.3, Geneva 
University Hospital and Technion}. \\
Available at \R2Lurl{http://stir.sourceforge.net }{http://stir.sourceforge.net}. 



 {[}Rea98] Reader A J, Visvikis A, Erlandsson K, Ott R J, and Flower 
M A \textbf{(1998)} Intercomparison of four reconstruction techniques 
for positron volume imaging with rotating planar detectors. \textit{Phys. 
Med. Biol.}, 43: 823-34.


{[}She82] Shepp L A and Vardi Y \textbf{(1982)} Maximum likelihood reconstruction 
for emission tomography. \textit{IEEE Trans. Med. Imaging}, 1: 113-122.


{[}Sil90] Silverman B W, Jones M C, Wilson J D, and Nychka D W \textbf{(1990)} 
A smoothed EM approach to indirect estimation problems, with 
particular reference to stereology and emission tomography. \textit{J. 
Roy. Stat. Soc}., 52: 271-324. 

{[}Thi06{]} Thielemans K, Mustafovic S, Tsoumpas C\textbf{(2006)},
STIR: Software for Tomographic Image
Reconstruction Release 2, 
\textit{Proc. IEEE Medical Imaging Conference 2006, San Diego, CA, USA.}

{[}Thi07{]} Thielemans K, Manjeshwar R, Tsoumpas C and Jansen F\textbf{(2007)}, 
A new algorithm for scaling of PET scatter estimates using all coincidence events. 
\textit { IEEE Nucl Sci Symp Med Imaging Conf } 3586 - 3590.

{[}Tso04{]} Tsoumpas C, Aguiar P, Nikita K S, Ros D and Thielemans K\textbf{(2004)}, 
Evaluation of the single scatter simulation algorithm implemented in the STIR library. 
\textit { IEEE Nucl Sci Symp Med Imaging Conf } 3361-3365.

{[}Tso07{]} Tsoumpas C, Turkheimer F E and Thielemans K  \textbf{(2007)}, 
Convergence properties of algorithms for direct parametric estimation of linear models in dynamic PET. 
\textit{IEEE Nucl Sci Symp Med Imaging Conf} 3034-3037

{[}Tso08{]} Tsoumpas C, Turkheimer F E and Thielemans K \textbf{(2008)}, Study of direct and indirect 
parametric estimation methods of linear models in dynamic positron emission tomography. \textit{Med. Phys.} 
35(4): p. 1299-1309.</i>


{[}Uns93] Unser M, Aldroubi A, Eden M \textbf{(1993)}. B-Spline Signal Proccessing: Part II - Efficient Design and Applications.  \textit{IEEE Trans Signal Processing} 41(2): 834-848.

{[}Uns99] Unser M. \textbf{(1999)}. Splines: A Perfect Fit for Signal and Image Processing. \textit{IEEE Signal Processing Magazine}: 22-38.

{[}Wat96{]} Watson C C, Newport D, and Casey M E\textbf{(1996)}, 
A single-scatter simulation technique for scatter correction in 3D PET. 
\textit { Fully 3D Image Recon Radiol Nucl Med} Kluwer Academic: 255-268. 

{[}Wat97a{]} Watson C C, Newport D, Casey M E, deKemp R A, Beanlands R S, and Schmand M 
\textbf{(1997)}, Evaluation of Simulation-Based Scatter Correction for 3-D PET Cardiac Imaging. 
\textit { IEEE Trans Nucl Sci }, 44(1): 90-97.

{[}Wat97b{]} Watson C C, Newport D, Casey M E, Beanlands R S, Schmand M, and deKemp R A 
\textbf{(1997)}, A technique for measuring the energy response of a PET tomograph using a compact scattering source. 
\textit { IEEE Trans Nucl Sci }, 44(6): 2500-2508.

{[}Wat00{]} Watson C C \textbf{(2000)}, New, faster, image-based scatter correction for 3D PET. 
\textit { IEEE Trans Nucl Sci } 47(4): 1587-1594. 

{[}Wat04{]} Watson C C, Casey M E, Michel C, Bendriem B\textbf{(2004)}, 
Advances in scatter correction for 3D PET/CT. \textit { IEEE Nucl Sci Symp Med Imaging Conf } 3008 - 3012. 

{[}Wer02{]} Werling A, Bublitz O, Doll J, Adam L E, and Brix G\textbf{(2002)}, 
Fast implementation of the single scatter simulation algorithm and its use in iterative image 
reconstruction of PET data. \textit { Phys Med Biol } 47(16): 2947-2960.

\end{document}
