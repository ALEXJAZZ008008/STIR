%&LaTeX
\batchmode
\documentclass{article}
\usepackage{a4wide}
\usepackage{setspace}
%\usepackage{graphicx}
\usepackage{hyperref}
\def\R2Lurl#1#2{\mbox{\href{#1}{\tt #2}}}
%\newcommand{\tab}{\hspace{5mm}}

\begin{document}

\begin{center}
\begin{spacing}{1.5}
\textbf{{\huge Upgrading from STIR version 1.4 to 2.0}}\\
\textbf{Kris Thielemans}
\end{spacing}
\end{center}

\tableofcontents 

\section{Overview}
This document summarises what the differences are
between STIR version 2 and 1.4.

The main change in STIR 2.0 is in the reconstruction code. 
All iterative reconstructions use now an objective function hierarchy. See 
the doxygen documentation for \texttt{GeneralisedObjectiveFunction}. This
will make it far easier to add different types of reconstruction 
algorithms, and in particular applicable to something else than PET
emission data. For instance, in STIR 2.0 there is now a facility to reconstruct
list mode data as well. In addition, reconstruction classes have now been
templated in the type of the ``target'' data, i.e. the parameters that 
need to be estimated. In STIR 1.4 this was fixed to images
(or in fact \texttt{DiscretisedDensity<3,float>} objects. 
The following describes what the consequences are for you.

\section{For Users}
Generally, all utilities work in the same way as before. There are
changes in some parameters used for parsing.

\subsection{Incompatible behaviour}
\begin{itemize}
\item Old OSMAPOSL parameter files no longer work. See the section below on OSMAPOSL,
   and sample files in the \texttt{samples} directory. Also, output is not necessarily
   identical. See section \ref{sec:Reconstruction_programs} for details.
\item The parsing keyword \texttt{output file format} in the reconstruction hierarchy
and \texttt{generate\_image} has been changed to \texttt{output file format type} to
be consistent with other parameters (it was already called that way in \texttt{stir\_math}.
\item The format for the parameter files for \texttt{generate\_image} and
   \texttt{list\_ROI\_values} changes somewhat. The only difference is the
   specification of the origin of the shape as a single keyword specifying \{z,y,x\}.
\item Image-origin is now read from and saved to file for those file formats that
    support it (at time of writing: Interfile, ECAT7 and ECAT6 for x,y info).
    As this was ignored before (origin was set to 0,0,0), this might mean that 
    the images are now forward projected/reconstructed differently.\\
    \textbf{WARNING:} At present, origin info is not set/preserved for
    projection data. This is a major deficiency and will be fixed at a later stage.
   At present, the first sinogram of segment 0 is supposed to be at \texttt{z=0} and
   the axis of the scanner at \texttt{x=y=0}.
\end{itemize}

\subsection{New utilities}
\begin{description}
\item[conv\_AVW]
If you have the \textit{AVW}\texttrademark{} library
\footnote{See \R2Lurl{http://www.mayo.edu/bir/Software/AVW/AVW1.html}{www.mayo.edu/bir/Software/AVW/AVW1.html}.
} installed on your system, the make process should have built \texttt{utilities/conv\_AVW}.
This utility allows to use the \textit{AVW} library to read an image, and then write it out
using STIR as Interfile.\\
\textbf{Warning:} the \textit{AVW} library seems to do flip some images depending
on the file format. For instance, it reads \textit{ECAT7} files using a z-flip compared
to STIR.
\item[SimSET support]
There are some preliminary files for make it easier to use SimSET and STIR together in the
\textbf{SimSET} directory. See the README.txt in that directory for more info.
\item[list\_image\_values, find\_maxima\_in\_image]
Utilities for getting data from images from the command line
\item[attenuation\_coefficients\_to\_projections, calculate\_attenuation\_coefficients]
Utilities for processing of attenuation coefficients.
\item[copy\_ecat7\_header]
Allows copying header info between ECAT7 files.
Note that all ECAT utilities are now in the \texttt{utilities/ecat} sub-directory.
\end{description}

\subsection{Filters \textit{aka} data processors}
The \texttt{ImageProcessor} hierarchy is now generalised to 
\texttt{DataProcessor<DataT>} that can work on different types than images. This is reflected
in the name of the parameters of some data processors.

\begin{description}
\item[Chained Image Processor]
is now renamed to \texttt{Chained Data Processor}. 
This data processor is used to allow you to specify two
data processors that will run in sequence. So, all its keywords have
been changed accordingly.
\begin{verbatim}
Chained Data Processor Parameters:=
Data Processor to apply first:= 
Data Processor to apply second:=
END Chained Data Processor Parameters:=
\end{verbatim}

\item[Threshold Min To Small Positive Value]
used to truncate to a cylindrical FOV. This facility no longer
exists. There is a new filter for this \texttt{Truncate To Cylindrical FOV}.
See also section \ref{sec:changes_in_OSMAPOSL_images}.
\end{description}


\subsection{Shapes}
The code to specify Euler angles is disabled
as it was buggy (get after set was not consistent).
You now have to  use a matrix to define direction vectors. 
(Note that in previous version the direction
vectors were ignored during parsing (this was a bug).)

The origin is now also specified with a single keyword.
\begin{verbatim}
; next keyword can be used for non-default axes
; values below are give a rotation around y for 90 degrees (swapping x and z)
; Warning: this uses the STIR convention {z,y,x}
direction vectors (in mm):= { {0,0,1}, {0,1,0}, {-1,0,0}}
; origin w.r.t. to standard STIR coordinate system (middle of first plane)
origin (in mm):= {z,y,x}
\end{verbatim}

\subsubsection{Box3D}
A contribution by C. Ross Schmidtlein and Assen S. Kirov. Allows
to specify 3D boxes.

\subsubsection{EllipsoidalCylinder}
A contribution by C. Ross Schmidtlein and Assen S. Kirov. Allows
to specify specify angle parameters to specify a wedge of the
cylinder.


\subsection{Reconstruction programs}
\label{sec:Reconstruction_programs}
Analytic reconstruction algorithms have not changed, except that there
is now a new keyword
\begin{verbatim}
post filter type:=
\end{verbatim}
However, as mentioned above, iterative reconstruction code has been 
changed dramatically. Because of this, the parameter format for 
\texttt{OSMAPOSL} has changed.
Previous .par files are incompatible. However, this is mostly a
question of reordering the keywords. Generally speaking, everything
related to the input data, sensitivity and the prior 
(except the \texttt{MAP model} keyword)
has now to be between
\begin{verbatim}
objective function type:= \
  PoissonLogLikelihoodWithLinearModelForMeanAndProjData
PoissonLogLikelihoodWithLinearModelForMeanAndProjData Parameters:=
; input, sensitivity and prior parameters here
end PoissonLogLikelihoodWithLinearModelForMeanAndProjData Parameters:=
\end{verbatim}
Also, the \texttt{sensitivity} program no longer exists. Its functionality
is now fully integrated into \texttt{OSMAPOSL} (and other forthcoming
algorithms). See below for details.
.
\subsubsection{Keyword name changes}
Some keywords changed because the \texttt{OSMAPOSL} code can now work
on different ``target types'', i.e. not only images.
\begin{description}
\item[sensitivity image] is now \texttt{sensitivity filename}.
\item[save image at subiteration intervals] is now
  \texttt{save estimate at subiteration intervals}.
\item[initial image] is now \texttt{initial estimate}.
\end{description}

\subsubsection{Keywords (and functionality) that are no longer present}
\begin{description}
\item[do rim truncation] is now effectively always set to 0. See also
section \ref{sec:changes_in_OSMAPOSL_images}.
\end{description}

\subsubsection{Changes for computing the sensitivity image}
In STIR 1.4, a separate executable \texttt{sensitivity} was used to compute
the sensitivity image. This is now a part of \texttt{OSMAPOSL} with 
relevant keywords (with defaults indicated)
\begin{verbatim}
time frame definition filename:=
time frame number:= 1
Bin Normalisation type:= None
sensitivity filename:=
recompute sensitivity := 0
\end{verbatim}
The first three are used to specify normalisation and attenuation factors 
as in \texttt{correct\_projdata}. 
See the User\'s Guide for more information.

\subsubsection{New functionality in OSMAPOSL}
There is a new keyword
\begin{verbatim}
use_subset_sensitivities
\end{verbatim}
which defaults to 0, which is the old behaviour. If set to 1, the denominator
term in OSMAPOSL uses sub-sensitivities as opposed to the total sensitivity image
(divided by the number of subsets). This avoids artefacts when using a larger
number of subsets and attenuation and/or normalisation in the model, at the 
expense of using more memory.
At time of writing, this has to be combined with
\begin{verbatim}
recompute sensitivity := 1 
\end{verbatim} 
The sub-sensitivities are not written to file.

\subsubsection{Changes in reconstructed images when using OSMAPOSL}
\label{sec:changes_in_OSMAPOSL_images}

In STIR 1.4, \texttt{OSMAPOSL} used ``rim truncation'' by default. This 
set all voxels outside a cylindrical FOV to zero. In version 2.x, this facility
no longer exists. The main reason for this is that the
\texttt{OSMAPOSLReconstruction} class now works for target types which are 
not images. In addition, you no longer have to jump through a lot of hoops 
if you want to use a square FOV. 

Also, a bug was removed that prevented\\
\texttt{enforce initial positivity condition:=1}\\
to work.

As a consequence, reconstructed images might not be identical. In the first
few iterations, the difference is only in the rim, but this then
induces differences in the whole image during subsequent iterations. Of course,
if you do see appreciable differences, it shows that there probably is something
wrong with your reconstructions, as most likely there should be no activity
in those rim voxels in any case. One potential reason is that you are
using randoms or scatter pre-corrected data, or do not use this corrections 
at all. The theoretically recommended, and most numerically stable,
 way is to use the \texttt{additive sinogram} parameter.\\


\paragraph{Recommendations}
\begin{itemize}
\item No longer use rim truncation. If you need a cylindrical FOV, let the
projectors do this for you (they do this by default).
\item If you see pixels at the edge of the FOV which are very high, you
probably can solve this by using  \texttt{additive sinogram} appropriately.
\end{itemize}

\paragraph{Guidelines on how to obtain identical reconstructions\\}
Read this section if you really need compatibility. This is a bit 
messy though, as there were some problems
with the rim truncation in STIR 1.4. So, we need to explain first
what happened.

Rim truncation was applied in 3 different processing steps:
\begin{itemize}
\item After reading the initial image, if
\texttt{enforce initial positivity condition} was set to 1 (the default).
\item After dividing the updated image with the sensitivity image, if
\texttt{do rim truncation} was set to 1 (the default).
\item After applying the inter-iteration or inter-update filter (but
only if they were set).
\end{itemize}
This first and third step were a (potentially unexpected) consequence of using
the image processor \texttt{ThresholdMinToSmallPositiveValueImageProcessor}. 

Unfortunately, the first and third step used a FOV that could be
a few pixels larger than the second step. This is 
because they
\footnote{Routines \texttt{threshold\_min\_to\_small\_positive\_value}
and \texttt{min\_positive\_value}.}
 used a test $x^2+y^2 <= R^2$, while the explicit rim
truncation
\footnote{Routines \texttt{divide\_and\_truncate} and \texttt{truncate\_rim}}
 used $x^2+y^2 < R^2$.
Finally, the origin used for this computation was shifted half a pixel in 
x and y for even-sized images w.r.t. the standard STIR origin.

The final piece of the puzzle is that the interpolating backprojector has a 
FOV which is slightly smaller than the rim truncation routines used in 
STIR 1.4.

Conclusions
\begin{itemize}
\item Reconstructions should be identical if \texttt{OSMAPOSL} in 
STIR 1.4 used\\
\texttt{
do rim truncation:=0\\
enforce initial positivity condition:=0
}\\
and there was no inter-update nor inter-iteration filtering.

\item Most likely, differences will be negligible if you use the interpolating
backprojector.

\item Otherwise, you can use the \texttt{Truncate To Cylindrical FOV}
data processor in STIR 2.0. Use it with
\begin{verbatim}
strictly_less_than_radius:=0
\end{verbatim}
to reproduce the first and third step in STIR 1.4 processing, and with
\begin{verbatim}
strictly_less_than_radius:=1
\end{verbatim}
for the second step. Note however, that if
\texttt{do\_rim\_truncation} was switched on, the 
\texttt{Truncate To Cylindrical FOV} filter would have to be used
as inter-iteration filter at every sub-iteration, which might not be 
possible if you are using another inter-iteration filter.\\
See the \texttt{OSMAPOSL\_test\_PM\_MRP.par}
and \texttt{postfilter\_truncate\_circular\_FOV.par} files in the 
\textbf{recon\_test\_pack} for examples. 
\end{itemize}


\subsubsection{Parallelisation of OSMAPOSL using MPI}
Contribution by Tobias Beisel. Allows running OSMAPOSL in master-slave mode.
See some more info in the User's Guide.
\subsubsection{Parallelisation of FBP2D using OPENMP}
Contribution by Tobias Beisel. Allows running FBP2D using threads.
However, this does not really speed-up the reconstruction. This code could
serve as a basis for FBP3DRP however.

\section{For Developers}

Before reading this, it is recommended that you read the section for
users first.

As mentioned before, the main changes are in the \texttt{Reconstruction}
hierarchy. This is now templated in \texttt{TargetT}. Moreover,
\texttt{IterativeReconstruction} now use an objective function hierarchy. See 
the doxygen documentation for \texttt{GeneralisedObjectiveFunction}. This
reorganisation has had repercusions in various other places. Examples are
\begin{itemize}
\item Class \texttt{OutputFileFormat} is now also templated to allow
other types than images (or in fact \texttt{DiscretisedDensity<3,float>}).
\item Class \texttt{ImageProcessor} is now replaced by the templated
class \texttt{DataProcessor<DataT>}.
\item Most functions in \texttt{buildblock/recon\_array\_functions.cxx}
were specific to images samples on cartesian grids 
(or even (\texttt{VoxelsOnCartesianGrid<float>} objects), and hence
could no longer be used. In addition, they had some ugly quircks
(see section \ref{sec:changes_in_OSMAPOSL_images}). So most of these
functions have now been removed.
\item Removed replace\_extension and add\_extension functions that work on char * as they were 
unsafe (in case the array did not have enough memory). Use std::string versions instead.

\end{itemize}

Other relatively important changes are listed below
\begin{itemize}
\item On request, the reconstruction hierarchy has now \texttt{get/set} functions for 
most parameters. Please email if some are missing.
\item There is now a new class \texttt{AnalyticReconstruction}.
\item \texttt{VectorWithOffset}: added
  constructors and 1 member to be able to have vectors that use existing
  data. This is useful when for example a C array with data needs to be
  manipulated.

\item \texttt{KeyParser}: 
  \begin{itemize} 
  \item use the 'parameter' member (of type
  boost::any) for storing all values, making par\_int, par\_double etc
  obsolete and enabling some nice use of macros to shorten the switch
  statements.  
  \item made IntVect and DoubleVect private typedefs (and removed
  UlongVect) 
  \item fix parameter\_info() for types that
  output to more than 1 line (such as 2D arrays). We now add the usual
  continuation character (a backslash), such that the output of
  parameter\_info can be parsed without problems.  
  \item in parameter\_info(),
  deleted an extra newline after calling parameter\_info() for a parsing
  object.
  \end{itemize}

\item \texttt{distributable}:
  no longer use global variables (for
  projectors etc) in distributable\_computation. This means (among other
  things) a better chance of thread-safety, although there's still work to
  do there.

\item in Reconstruction hierarchy:  moved most checks
  from post\_processing() to set\_up() and clarified in doxygen which function 
  is for what purpose

\item \texttt{DiscretisedDensity}
 \begin{itemize}
  \item renamed get\_empty\_discretised\_density() to get\_empty\_copy(), using 
  covariant return types 
  \item added has\_same\_characteristics() 
  \item added functions to convert between \textit{physical} coordinates (in mm)
    and index coordinates.
  \end{itemize}

\item New global function \texttt{read\_from\_file()} using an input file format 
registry. This makes it much easier to add support for a new file format. See
class \texttt{InputFileFormatRegistry}.

\item New variables defined for \texttt{make} for \textit{AVW} support:
\texttt{AVW\_INCLUDE\_DIR}, \texttt{AVW\_LIBS} and \texttt{HAVE\_AVW}. They
are set in \texttt{config.mk} according to recommend \textit{AVW} values,
i.e. using the environment variables \texttt{AVW} and \texttt{TARGET}.
See the Users Guide.
\end{itemize}

In addition, I have taken the opportunity to do some other bug fixing.
The main affected area is the \texttt{Shape3D} hierarchy, where the 
Euler code had to be disabled and a different mechanism is now used to
spcify direction vectors. This means that most of this code is now
ready for templating in the number of dimensions. For other smaller
changes, please refer to the ChangeLog.

\subsection{Additional new functionality}
This list is incomplete. Please refer to the ChangeLog.
\begin{itemize}
\item added \texttt{has\_same\_characteristics()} to \texttt{Sinogram}, \texttt{Viewgram}, 
       \texttt{RelatedViewgrams}, \texttt{Segment*}
\item added \texttt{operator==} to \texttt{Sinogram}, \texttt{Viewgram}, 
       \texttt{RelatedViewgrams}, \texttt{Segment*}, \texttt{DiscretisedDensity} 
       and symmetry classes. Updated \texttt{operator==} for \texttt{ProjDataInfo}
       classes to check also fields specific to the leaf classes.
\item added \texttt{find\_centre\_of\_gravity\_in\_mm} for \texttt{VoxelsOnCartesianGrid} objects.
\end{itemize}

\subsection{Additional incompatibilities}
This list is incomplete. Please refer to the ChangeLog.
\begin{itemize}
\item The \texttt{zoom\_image} functions have now much better defined behaviour
	w.r.t the new index ranges and the origin (see the doxygen). In particular,
	the z-range now starts from index 0, and the origin is updated such that
	it remains physically correct. Another incompatibility is that previously, 
	the multiple argument versions of \texttt{zoom\_image} (i.e. specifying zoom etc 
	explicitly) modified the image in place. Now, the function with the same name 
	returns a new image instead. The previous functionality is now provided by 
	\texttt{zoom\_image\_in\_place}.

\item \texttt{find\_centre\_of\_gravity\_in\_mm\_per\_plane} now returns a result that takes 
    \texttt{image.get\_origin()} into account properly. It no longer
    shifts the z-direction to the centre of the image. There is currently no
    utility that depends on this behaviour though.
\end{itemize}
\end{document}
